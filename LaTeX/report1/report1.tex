\documentclass{article}

% Latex Tutorial beginners: http://www.latex-tutorial.com/tutorials/beginners/
\newcommand{\comment}[1]{}

\newcommand{\seq}[0]{$P_{s}$}
\renewcommand{\mp}[0]{$P_{m}$}
\newcommand{\ndp}[0]{$P_{np}$}
\newcommand{\note}[1]{{\tiny (#1)}}

\newcommand{\algo}[0]{Split\&Merge-Segmentation}

% Packages explained - Adding more functions to LATEX http://www.latex-tutorial.com/tutorials/beginners/lesson-3/
\usepackage[utf8]{inputenc}     % Zum verwenden von ä,ü und ö: http://en.wikibooks.org/wiki/LaTeX/Special_Characters
\usepackage{booktabs}
\usepackage{url}
\usepackage{hyperref}

\usepackage[margin=1.3in]{geometry}

% Erzeugen des PDF (ohne References):
% Strg + Alt + 1 in Gedit

% Erzeugen des PDF (mit References):
% Strg + Alt + 1 in Gedit
% $ bibtex expose
% Zweimal: Strg + Alt + 1 in Gedit

% fügt die Literatur auch zum Inhaltsverzeichnis hinzu (ist so nicht default)
\usepackage[nottoc,notlot,notlof,numbib]{tocbibind}

% Adding a bibliography http://www.latex-tutorial.com/tutorials/beginners/lesson-7/
\bibliographystyle{apalike}


% Nested numbered enumerations, eg. 1.1, 1.2: http://tex.stackexchange.com/questions/78842/nested-enumeration-numbering
\renewcommand{\labelenumii}{\theenumii}
\renewcommand{\theenumii}{\theenumi.\arabic{enumii}.}


\title{
    Progress Report 1 \\[7pt]
    \large Nested Data Parallelism for\\ Image Processing Algorithms
}
\date{06.05.2015}
\author{Chandrakant Swaneet Kumar Sahoo}


\begin{document}

  \pagenumbering{arabic}
  
  \maketitle
  
  
  \section{Leading question}
  \paragraph{How can we implement this high-level irregularly-parallel algorithm such that it compiles to efficient mashine level code?}
  \paragraph{Subquestions}
      \begin{itemize}
      \item How much faster is a parallel variant against the sequential one?
      \item How much faster is a compiled variant to human-written low-level parallel code?
      \end{itemize}
  
  
  \section{What I have done}
    I have read further papers on how Nested Data Parallel works.
    I have also decided to manually transform/vectorize the program, because
    the generated program is hard to read/too long... and sometimes GHC can't compile.
    \note{NDP is still work in progress!}
    
    I have chosen \href{http://en.wikipedia.org/wiki/Histogram_equalization#Full-sized_image}{Histogram-Balancing} as my algorithm.
    The algorithm has sufficient opportunities to demonstrate the powers of NDP and
    is small enough to be manually transformed - as opposed to ShortestPaths, Connected Components Labeling, etc...
    
  \section{What I will do next}
    I will test my implementation and complete the transformations.
    This will take a while. Then I am ready to do the other two (easier) implementations \seq  and \mp.

  \section{Progress}
    \begin{itemize}
      \item[80\%] Read more papers on Nested Data Parallel Haskell
      \item[60\%] Read more papers on Analysis of Parallel Progrmas
      \item[100\%] Decide on an algorithm (Histogram Balancing)
      \item[50\%] Program Transformation:
        \begin{itemize}
          \item[100\%] Desugar
          \item[50\%] Vectorization
          \item[0\%] Inlining \& Fusioning \note means Optimization
        \end{itemize}
      \item[0\%] Implement sequential variant
      \item[0\%] Implement manually-parallelized variant
      \item[0\%] Answer first subquestion
      \item[0\%] Answer second subquestion
      \item[?\%] \textit{stuff}
      \item[0\%] Written thesis
      \item[0\%] Colloquium
    \end{itemize}
    
    \section{Time Table}
    \begin{table}[h]
        \begin{center}
        \caption{Time table} % http://www.latex-tutorial.com/tutorials/beginners/lesson-8/
        \label{timetable}
        \begin{tabular}{lrrl}
            \toprule
            Current Week & CW & monday & thesis work \\
            \midrule
                & 17 & 20.04 & reading remaining papers, reading parallel complexity theory \\
                & 18 & 27.04 & deciding on an algorithm  \\
            now & 19 & 4.05  & \textit{implementing \ndp, vectorizing and optimizing \ndp} \\
                & 20 & 11.05 & implementing \ndp, vectorizing and optimizing \ndp\\
                & 21 & 18.05 & implementing \seq and \mp \\
                & 22 & 25.05 & analysis \& comparision \\
                & 23 & 1.06  & \textit{puffer} \\
                & 24 & 8.06  & \textit{puffer} \\
                & 25 & 15.06 & Begin to write down, prepare for exams\\
                & 26 & 22.06 & Writing..., prepare for exams \\
                & 27 & 29.06 & Writing..., prepare for exams \\
                & 28 & 6.07  & Prepare Colloquium, Writing..., exams week 1 \\
                & 29 & 13.07 & Prepare Colloquium, Finalize writing, exams week 1 \\
                & 30 & 20.07 & Colloquium and Release \\
                & 31 & 27.07 & Last week for Colloquium and Release \\
                & 32 & 3.08  & Fin \texttt{:D} \\
        \end{tabular}
        \end{center}
    \end{table}
    
\end{document}

