\documentclass{article}

% Latex Tutorial beginners: http://www.latex-tutorial.com/tutorials/beginners/


% Packages explained - Adding more functions to LATEX http://www.latex-tutorial.com/tutorials/beginners/lesson-3/
\usepackage[utf8]{inputenc}     % Zum verwenden von ä,ü und ö: http://en.wikibooks.org/wiki/LaTeX/Special_Characters
\usepackage{booktabs}
\usepackage{url}
\usepackage{hyperref}
\usepackage[T1]{fontenc}
\usepackage[margin=1.3in]{geometry}
\usepackage[nottoc,notlot,notlof,numbib]{tocbibind}
\usepackage{listings,xcolor}
\usepackage{amsmath}


\newcommand{\comment}[1]{}

\newcommand{\seq}[0]{$P_{seq}$}
\newcommand{\man}[0]{$P_{man}$}
\newcommand{\ndpn}[0]{$P_{nest}$}
\newcommand{\ndpv}[0]{$P_{vect}$}
\newcommand{\algo}[0]{Histogram Balancing}
\newcommand{\type}[1]{\texttt{#1}}
\newcommand{\ts}[0]{\,}% type space
\newcommand{\Hist}[1]{[:\ts#1\ts:]}
\newcommand{\pa}[1]{[:#1:]}

\newcommand{\eqdo}[0]{\hskip 4em}
\newcommand{\W}[0]{\textrm{W}}
\newcommand{\D}[0]{\textrm{D}}

\definecolor{keywordblue}{HTML}{5F49CC}
\lstset{
  frame=none,
  xleftmargin=2pt,
  belowcaptionskip=\bigskipamount,
  captionpos=b,
  escapeinside={*'}{'*},
  language=haskell,
  tabsize=2,
  gobble=6,
  emphstyle={\bf},
  commentstyle=\it,
  stringstyle=\mdseries\rmfamily,
  showspaces=false,
  columns=flexible,
  showstringspaces=false,
  morecomment=[l]\%,
  basicstyle=\ttfamily,
  keywordstyle=\color{keywordblue},
  keywords={type,if,then,else,let,in}
}

\begin{document}

    \section{ \seq}
      \begin{lstlisting}
      type Image  = V.Vector (V.Vector Int)
      type Hist a = Map Int a

      hbalance :: Image -> Image
      hbalance img =
        let h = hist img
            a = accu h
            a0 = M.first a      -- M.function refers to functions
            agmax = M.last a    -- of the tree-based Map datastructure
            n = normalize a0 agmax a
            s = scale gmax n
            img' = apply s img
        in img'

      hist :: Image -> Hist Int   -- whereas V.function refers to dense-array Vector datastructure
      hist = V.foldr (\i -> M.insertWith (+) i 1) M.empty . V.concat

      accu :: Hist Int -> Hist Int
      accu = M.scanl (+) 0

      normalize :: Int -> Int -> Hist Int -> Hist Double
      normalize a0' agmax' as =
          let a0 = fromIntegral a0'
              agmax = fromIntegral agmax'
              divisor = agmax - a0
          in  M.map (\freq' -> (fromIntegral freq' - a0) / divisor) as

      scale :: Int -> Hist Double -> Hist Int
      scale gmax = M.map (\d -> floor (d * fromIntegral gmax))

      apply :: Hist Int -> Image -> Image
      apply as img = V.map (V.map (M.lookupLessEqual as)) img
      \end{lstlisting}
    
    \newpage
    
    \section{Work and Depth Table}
      
      \begin{itemize}
        \item n sei die Anzahl der Bildpixel
        \item w sei die Bildbreite
        \item h sei die Bildhöhe
        \item p sei die Anzahl der PUs (gang members).
      \end{itemize}
      
      \paragraph{}
        \begin{table}[h]
          \caption{Work and Depth complexities}
          \label{timetable}
          \begin{tabular}{lll}
              \toprule
              function or variable &  O(W) and O(D) \\
              \midrule
              hbalance          & max(n * log gmax, gmax) \\
              \midrule
              hist              & n * log gmax\\
              V.concat          & n\\ 
              M.insertWith      & log gmax\\ 
              V.foldr           & n * log gmax \\ 
              \midrule
              accu              & gmax\\ 
              M.scanl           & gmax\\
              \midrule
              normalize         & gmax\\ 
              scale             & gmax\\ 
              M.map             & gmax\\ 
              \midrule
              apply             & n * log gmax = w * h * log gmax \\
              M.lookupLessEqual & log gmax \\ 
              \midrule
              M.empty           & 1\\ 
              M.first           & 1\\ 
              M.last            & 1\\
              \midrule
              V.map f xs        & size(xs)*W(f,x)\\ 
          \end{tabular}
        \end{table}
      
    \section{Other aspects \small{e.g. sync-points, programmer workload, simplicity}}
      \begin{itemize}
        \item optimisations: sequenial stream fusion is applied to the array operations here
            automatically. However, the program is entirely sequential
        \item progammer-workload: written in less than 1 hour
        \item simplicity: straightforward implementation from a book
      \end{itemize}
      

\end{document}



