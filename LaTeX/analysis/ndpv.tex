\documentclass{article}

% Latex Tutorial beginners: http://www.latex-tutorial.com/tutorials/beginners/


% Packages explained - Adding more functions to LATEX http://www.latex-tutorial.com/tutorials/beginners/lesson-3/
\usepackage[utf8]{inputenc}     % Zum verwenden von ä,ü und ö: http://en.wikibooks.org/wiki/LaTeX/Special_Characters
\usepackage{booktabs}
\usepackage{url}
\usepackage{hyperref}
\usepackage[T1]{fontenc}
\usepackage[margin=1.3in]{geometry}
\usepackage[nottoc,notlot,notlof,numbib]{tocbibind}
\usepackage{listings}
\usepackage{amsmath}


\newcommand{\comment}[1]{}

\newcommand{\seq}[0]{$P_{seq}$}
\newcommand{\man}[0]{$P_{man}$}
\newcommand{\ndpn}[0]{$P_{nest}$}
\newcommand{\ndpv}[0]{$P_{vect}$}
\newcommand{\algo}[0]{Histogram Balancing}
\newcommand{\type}[1]{\texttt{#1}}
\newcommand{\ts}[0]{\,}% type space
\newcommand{\Hist}[1]{[:\ts#1\ts:]}
\newcommand{\pa}[1]{[:#1:]}

\newcommand{\eqdo}[0]{\hskip 4em}
\newcommand{\W}[0]{\textrm{W}}
\newcommand{\D}[0]{\textrm{D}}

% Adding a bibliography http://www.latex-tutorial.com/tutorials/beginners/lesson-7/
\bibliographystyle{apalike}


\title{Costs for the algorithm}
\date{19.05.2015}
\author{Chandrakant Swaneet Kumar Sahoo}


\lstset{
  frame=none,
  xleftmargin=2pt,
  belowcaptionskip=\bigskipamount,
  captionpos=b,
  escapeinside={*'}{'*},
  language=haskell,
  tabsize=2,
  gobble=6,
  emphstyle={\bf},
  commentstyle=\it,
  stringstyle=\mdseries\rmfamily,
  showspaces=false,
  keywordstyle=\bfseries\rmfamily,
  columns=flexible,
  basicstyle=\small\sffamily,
  showstringspaces=false,
  morecomment=[l]\%,
}

\begin{document}

    % \maketitle
    
    \section{The algorithm}
      \begin{lstlisting}
        type Image = PA (PA Int)
        type Hist  = PA Int

        hbalance :: Image -> Image
        hbalance img =
          let a :: Hist                                       -- Histogram
              a = joinD                                       -- accu end
                  . mapD (\(as,a) -> mapS (plusInt a) as)
                  . propagateD plusInt 0
                  . mapD (scanlS plusInt 0)                   -- accu begin
                  . sparseToDenseD (plusInt gmax 1) 0         -- hist end
                  . splitSparseD (plusInt gmax 1)
                  . joinD
                  . mapD tripletToATup2
                  . segdSplitMerge 0
                  . sortPS
                  . concatPS                                  -- hist begin
                  $ img                                       -- 1
                      
              n :: Int
              n = lengthPS a                                  -- 2
              
              a0, divisor, gmax' :: Double
              a0      = int2Double . headPS $ a               -- 3, variables for normalize and scale
              divisor = minusDouble (int2Double (lastPS a)) . int2Double . headPS $ a
              gmax'   = int2Double $ gmax
              
              normScale :: Int -> Int                         -- 0, body of normalize and scale
              normScale = floorDouble . (flip multDouble) gmax' . (flip divDouble) divisor . (flip minusDouble) a0 . int2Double
              
              as :: Hist                                      -- final mapping array
              as = joinD . mapD (mapS normScale) . splitD $ a -- 4, normalize and scale applied
              
              pixelReplicate :: Hist -> PA Hist               -- 0, artifact of NDP
              pixelReplicate = concatPS . replPL (lengths (getSegd xs)) . replPS (lengthPS img)
              
          in unconcatPS img
             . indexPL (pixelReplicate as)                    -- 5, apply. core of nested data parallelism here!
             . concatPS
             $ img
      \end{lstlisting}
    
    \newpage
    
    \section{Work and Depth Table}
      
      \begin{itemize}
        \item n sei die Anzahl der Bildpixel
        \item w sei die Bildbreite
        \item h sei die Bildhöhe
        \item p sei die Anzahl der PUs (gang members).
      \end{itemize}
      
      \paragraph{}
        \begin{table}[h]
          \begin{center}
          \caption{Work and Depth complexities}
          \label{timetable}
          \begin{tabular}{lll}
              \toprule
              function or variable &      O(W)           & O(D) \\
              \midrule
              hbalance        & \max(n \log n, gmax)& \log \max(n, gmax) \\
              hist            & \max(n \log n, gmax)& \log n \\
              accu            & gmax                & \log gmax \\
              as              & gmax                & 1\\
              pixelReplicate  & 1 (bzw. $n \cdot gmax$) & 1 \\
              img'            & n                   & 1 \\
              \midrule
              sortPS          & $n \log n$         & $\log n$ \\
              concatPS        & 1                   & 1 \\
              unconcatPS      & 1                   & 1 \\
              headPS,lastPS   & 1                   & 1 \\
              replPS n x      & n                   & 1 \\
              replPL ns xs    & sum(ns)*size(xs)    & 1 \\
              indexPL is xs   & size(xs)            & 1 \\
              \midrule
              mapD tripletToATup2  & gmax $\frac{p}{p}$ & 1 \\
              segdSplitMerge  & $n \log n$         & $\log n$ \\
              \midrule
              mapD fS xs      & $\W(fS,s)*size(xs)$ & $\W(fS,s)*size(xs)/p$ \\
              lengthPS        & 1                   & 1 \\
              joinD xs        & $size(xs)$          & 1 \tiny (communication overhead?) \\
              splitD xs       & $size(xs)$          & 1 \tiny (communication overhead?) \\
              propagateD xs   & 1                   & $\log size(xs)$ \\
              sparseToDenseD  & gmax                & 1 \\
              splitSparseD    & gmax                & 1 \\
          \end{tabular}
          \end{center}
        \end{table}
        
    \section{Calculating "hbalance" entirely}
      Let $n = |img| = w\cdot h$
      \begin{equation}
      \begin{split}
      \W(w \times h,gmax)
            & = \W(hist) + \W(accu) + \W(as) + \W(pixelReplicate) + \W(img') \\
            & = O( \max(n \log n, gmax) + gmax + gmax + 1 + n) \\
            & \in O(\max(n \log n, gmax)) \\
      \D(w \times h,gmax)
          & = \max \{ hist, accu, as, pixelReplicate, img'\} \\
          & = \max \{\log n, \log gmax \} \\
          & \in O(\log \max(n,gmax)) \\
      \end{split}
      \end{equation}
        
    \section{Calculating the histogram "hist"}
      Let $n = |img| = w\cdot h$
      \begin{equation}
      \begin{split}
      \W(w \times h)
            & = \W(concatPS) + \W(sortPS) + \W(segdSplitMerge) \\
            & + \W(mapD,localSegdToTup2) + \W(joinD) \\
            & = 1 + n \log n + n \log n + gmax + gmax \\
            & = 1 + 2 (n \log n + gmax) \\
            & \in O(\max(n \log n, gmax)) \\
      \D(w \times h)
            & = \max \{ concatPS,sortPS,...,joinD\} \\
            & \in O(\log n) \\
      \end{split}
      \end{equation}
    
    \section{Calculating the accumulated histogram "accu"}
      \begin{equation}
      \begin{split}
      \W(gmax)
            & = \W(splitSparseD) + \W(sparseToDenseD) + \W(mapD,scanlS) \\
            &     + \W(propagateD) + \W(mapD,mapS,plusInt) + \W(joinD) \\
            & = gmax + gmax + gmax + 1 + gmax + gmax\\
            & = 1 + 5 \cdot gmax \\
            & \in O(gmax) \\
      \D(gmax)
          & = \max \{ splitSparseD, sparseToDenseD, (mapD,scanlS),...\} \\
          & \in O(\log gmax)
      \end{split}
      \end{equation}
    
    \section{Calculating the mapping array "as"}
      \begin{equation}
      \begin{split}
      \W(gmax)
            & = \W(joinD) + \W(mapD,normScale) + \W(splitD) \\
            & = gmax + gmax + gmax \\
            & = 3 \cdot gmax \\
            & \in O(gmax) \\
      \D(gmax)
          & = \max \{ joinD,(mapD,normScale),splitD\} \\
          & \in O(1)
      \end{split}
      \end{equation}
      
    \section{Calculating by "pixelReplicate"}
      PixelReplicate is a function which executed strictly
      and literally would have a work complexity of $n \cdot gmax$.
      However, due to strong improvements in [WorkEfficient]
      this replication is simply cached locally such that
      the work complexity for this case is negilible and
      can be simplified to $1$.
      
      \begin{equation}
      \begin{split}
      \W(gmax,w \times h)
            & \in O(1) \\
      \D(gmax, w \times h)
            & \in O(1) \\
      \end{split}
      \end{equation}
      
    \section{Calculating the final balanced image "img'"}
      \begin{equation}
      \begin{split}
      \W(gmax, w \times h)
            & = \W(unconcatPS) + \W(indexPL) + \W(concatPS) \\
            & = 1 + n + 1 \\
            & \in O(n) \\
      \D(gmax, w \times h)
          & = \max \{ unconcatPS, indexPL, concatPS \} \\
          & \in O(1)
      \end{split}
      \end{equation}
      
    \section{Other aspects \small{e.g. sync-points, programmer workload, simplicity}}
      \begin{itemize}
        \item optimisations:
          removed sync-points(more local operations),
          tight distributed normalisation loop,
          distributed optimal prefix-sum,
          distributed groupP and sortP,
          nested data parallelism enables a depth lower than O(width)
        \item progammer-workload: as much as for \ndpn
        \item simplicity: many details, not indented for human readers, but shows optimisations clearly
      \end{itemize}
      

\end{document}



