\documentclass{article}

% Latex Tutorial beginners: http://www.latex-tutorial.com/tutorials/beginners/


% Packages explained - Adding more functions to LATEX http://www.latex-tutorial.com/tutorials/beginners/lesson-3/
\usepackage[utf8]{inputenc}     % Zum verwenden von ä,ü und ö: http://en.wikibooks.org/wiki/LaTeX/Special_Characters
\usepackage{booktabs}
\usepackage{url}
\usepackage{hyperref}
\usepackage[T1]{fontenc}
\usepackage[margin=1.3in]{geometry}
\usepackage[nottoc,notlot,notlof,numbib]{tocbibind}
\usepackage{listings,xcolor}
\usepackage{amsmath}


\newcommand{\comment}[1]{}

\newcommand{\seq}[0]{$P_{seq}$}
\newcommand{\man}[0]{$P_{man}$}
\newcommand{\ndpn}[0]{$P_{nest}$}
\newcommand{\ndpv}[0]{$P_{vect}$}
\newcommand{\algo}[0]{Histogram Balancing}
\newcommand{\type}[1]{\texttt{#1}}
\newcommand{\ts}[0]{\,}% type space
\newcommand{\Hist}[1]{[:\ts#1\ts:]}
\newcommand{\pa}[1]{[:#1:]}

\newcommand{\eqdo}[0]{\hskip 4em}
\newcommand{\W}[0]{\textrm{W}}
\newcommand{\D}[0]{\textrm{D}}

\definecolor{keywordblue}{HTML}{5F49CC}
\lstset{
  frame=none,
  xleftmargin=2pt,
  belowcaptionskip=\bigskipamount,
  captionpos=b,
  escapeinside={*'}{'*},
  language=haskell,
  tabsize=2,
  gobble=6,
  emphstyle={\bf},
  commentstyle=\it,
  stringstyle=\mdseries\rmfamily,
  showspaces=false,
  columns=flexible,
  showstringspaces=false,
  morecomment=[l]\%,
  basicstyle=\ttfamily,
  keywordstyle=\color{keywordblue},
  keywords={type,if,then,else,let,in}
}

\begin{document}

    \section{ \man}
      \begin{lstlisting}
        type Image  = V.Vector Int   -- unboxed vector. aka dense heap array
        type Hist   = V.Vector Int   -- the index is the gray-value. its value the result

        hbalance :: Image -> Image
        hbalance img =
          let hist = parAccuHist img
              min = hist ! 0
              max = hist ! gmax
              apply hist = parMap (\i -> h ! i) img
              sclNrm :: Int -> Int
              sclNrm x = round( (x-min)/(max - min)*gmax )
          in  apply (parMap sclNrm) hist

        parAccuHist :: Image -> Hist
        parAccuHist []  = replicate gmax 0            -- creates [0,...,0]
        parAccuHist [x] = generate gmax (\i -> if (i >= x) then 1 else 0 ) -- create [0,...0,1,...,1]
        parAccuHist xs  =
          let (left,right) = splitMid xs
              [leftRes,rightRes] = parMap parAccuHist [left,right] -- two parallel recursive calls
          in  parZipWith (+) leftRes rightRes
      \end{lstlisting}
    
    \newpage
    
    \section{Work and Depth Table}
      
      \begin{itemize}
        \item n sei die Anzahl der Bildpixel
        \item w sei die Bildbreite
        \item h sei die Bildhöhe
        \item p sei die Anzahl der PUs (gang members).
      \end{itemize}
      
      \paragraph{}
        \begin{table}[h]
          \caption{Work and Depth complexities}
          \label{timetable}
          \begin{tabular}{lll}
              \toprule
              function or variable & O(W)     & O(D) \\
              \midrule
              hbalance        & n * gmax      & log n \\
              apply           & n           & 1 \\
              parMap sclNrm   & gmax        & 1 \\
              parAccuHist     & n * gmax      & log n \\
              \midrule
              parAccuHist     & n * gmax      & log n \\
              splitMid        & 1           & 1 \\
              parZipWith      & gmax        & 1 \\
              replicate       & gmax        & 1 \\
              generate        & gmax        & 1 \\
              parMap f xs     & W(f,x)*size(xs) & 1 \\
              arr ! i          & 1           & 1 \\
          \end{tabular}
        \end{table}
        
    \section{Calculating "hbalance"}
      \begin{equation}
      \begin{split}
      \W(n,gmax)
            & = \W(parAccuHist) + \W(parMap,sclNrm) + \W(apply) \\
            & = n \cdot gmax + gmax + n \\
            & \in O(n \cdot gmax) \\
      \D(n,gmax)
          & = \max \{ parAccuHist, (parMap,sclNrm), apply\} \\
          & = \max \{ \log n, 1, 1\} \\
          & \in O(\log n) \\
      \end{split}
      \end{equation}
        
    \section{Calculating "parAccuHist"}
      On each recursive call of parAccuHist, there are consantly
      many calls to functions of work O(gmax) and depth O(1).
      The only exception are the two recursive calls.
      They call the problem with array of half-size.
      The depth of this function is logarithmic to the input size,
      because of this type of recursive calls.
      (If gmax is treated as a constant, then the following result
      can also be derived from the Master Theorems first case.)
      \begin{equation}
      \begin{split}
      \W(n,gmax)
            & = \begin{cases}
                 gmax & \text{ if } n \le 1 \\ 
                 2 \W(\frac{n}{2}) + gmax & \text{ else }
                \end{cases} \\
            & = \begin{cases}
                  gmax & \text{ if } n \le 1 \\ 
                  gmax2^0 + 2^1\W(\frac{n}{2}) & \text{ if } n = 2 \\
                  gmax2^0 + gmax2^1 + 2^2\W(\frac{n}{4}) & \text{ if } n = 3 \\
                  gmax2^0 + gmax2^1 + ... + gmax 2^{\log n - 1} + 2^{\log n}\W(1) & \text{ else }
                \end{cases} \\
            & = \textrm{ (... tying the knot ...) } \\
            & = gmax \sum_{i=0}^{\log n}{2^i} \\
            & = gmax (2^{\log n + 1} - 1) \\
            & = gmax (2n - 1) \\
            & \in O(n \cdot gmax) \\
      \D(n,gmax)
          & \in O(\log n) \\
      \end{split}
      \end{equation}
    \section{Other aspects \small{e.g. sync-points, programmer workload, simplicity}}
      \begin{itemize}
        \item optimisations:
          ?
        \item progammer-workload: ?
        \item simplicity: ?
      \end{itemize}
      

\end{document}



