\documentclass{article}

% Latex Tutorial beginners: http://www.latex-tutorial.com/tutorials/beginners/


% Packages explained - Adding more functions to LATEX http://www.latex-tutorial.com/tutorials/beginners/lesson-3/
\usepackage[utf8]{inputenc}     % Zum verwenden von ä,ü und ö: http://en.wikibooks.org/wiki/LaTeX/Special_Characters
\usepackage{booktabs}
\usepackage{url}
\usepackage{hyperref}
\usepackage[T1]{fontenc}
\usepackage[margin=1.3in]{geometry}
\usepackage[nottoc,notlot,notlof,numbib]{tocbibind}
\usepackage{listings,xcolor}
\usepackage{amsmath}


\newcommand{\comment}[1]{}

\newcommand{\seq}[0]{$P_{seq}$}
\newcommand{\man}[0]{$P_{man}$}
\newcommand{\ndpn}[0]{$P_{nest}$}
\newcommand{\ndpv}[0]{$P_{vect}$}
\newcommand{\algo}[0]{Histogram Balancing}
\newcommand{\type}[1]{\texttt{#1}}
\newcommand{\ts}[0]{\,}% type space
\newcommand{\Hist}[1]{[:\ts#1\ts:]}
\newcommand{\pa}[1]{[:#1:]}

\newcommand{\eqdo}[0]{\hskip 4em}
\newcommand{\W}[0]{\textrm{W}}
\newcommand{\D}[0]{\textrm{D}}

\definecolor{keywordblue}{HTML}{5F49CC}
\lstset{
  frame=none,
  xleftmargin=2pt,
  belowcaptionskip=\bigskipamount,
  captionpos=b,
  escapeinside={*'}{'*},
  language=haskell,
  tabsize=2,
  gobble=6,
  emphstyle={\bf},
  commentstyle=\it,
  stringstyle=\mdseries\rmfamily,
  showspaces=false,
  columns=flexible,
  showstringspaces=false,
  morecomment=[l]\%,
  basicstyle=\ttfamily,
  keywordstyle=\color{keywordblue},
  keywords={type,if,then,else,let,in}
}

\begin{document}

    \section{ \ndpn }
      \begin{lstlisting}
        type Image = [:[:Int:]:]
        type Hist a = [:a:]

        hbalance :: Image -> Image
        hbalance img =
            let h = hist img
                a = accu h
                a0 = headP a
                agmax = lastP a
                n = normalize a0 agmax a
                s = scale gmax n
                img' = apply s img
            in  img'

        hist :: Image -> Hist Int
        hist = 
            sparseToDenseP (gmax+1) 0
            . mapP (\g -> (headP g,lengthP g))
            . groupP
            . sortP
            . concatP

        accu :: Hist Int -> Hist Int
        accu = scanlP (+) 0

        normalize :: Int -> Int -> Hist Int -> Hist Double
        normalize a0' agmax' as =
            let a0 = fromIntegral a0'
                agmax = fromIntegral agmax'
                divisor = agmax - a0
            in  [: (fromIntegral freq' - a0) / divisor | freq' <- as :]

        scale :: Int -> Hist Double -> Hist Int
        scale gmax as = [: floor (a * fromIntegral gmax) |  a <- as :]

        apply :: Hist Int -> Image -> Image
        apply as img = mapP (mapP (as !:)) img
      \end{lstlisting}
    
    \newpage
    
    \section{Work and Depth Table}
      
      \begin{itemize}
        \item n sei die Anzahl der Bildpixel
        \item w sei die Bildbreite
        \item h sei die Bildhöhe
        \item p sei die Anzahl der PUs (gang members).
      \end{itemize}
      
      \paragraph{}
        \begin{table}[h]
          \caption{Work and Depth complexities}
          \label{timetable}
          \begin{tabular}{lll}
              \toprule
              function or variable &      O(W)           & O(D) \\
              \midrule
              hbalance        & \max(n \log n, gmax) & \log \max(n, gmax) \\
              \midrule
              hist            & \max(n \log n, gmax) & log n \\
              sparseToDenseP  & gmax                 & 1 \\
              groupP          & n                    & \log n \\
              sortP           & n \log n             & \log n \\
              concatP         & 1                    & 1 \\
              \midrule
              accu            & gmax                 & \log gmax \\
              scanlP          & gmax                 & \log gmax \\
              \midrule
              normalize       & gmax                 & 1 \\
              scale           & gmax                 & 1 \\
              \midrule
              apply           & $n = w \cdot h \cdot O(1)$ & 1 \\
              mapP f xs       & $W(f,x) \cdot size(xs)$      & 1 \\
              headP/lastP     & 1                    & 1 \\
              indexP, !:      & 1                    & 1 \\
          \end{tabular}
        \end{table}
      
    \section{Other aspects \small{e.g. sync-points, programmer workload, simplicity}}
      \begin{itemize}
        \item optimisations:
          ?
        \item progammer-workload: ?
        \item simplicity: ?
      \end{itemize}
      

\end{document}



