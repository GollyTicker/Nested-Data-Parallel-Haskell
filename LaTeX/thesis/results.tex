
% Vertrauen ist gut, Kontrolle ist besser.

This chapter summarises results of the four implementations
\seq, \man, \ndpn and \npdv. First the complexities of
the programs are given and observations
are made an discussed.
Then, the pro and contra of each program are given - 
comparing them on different aspects.
Finally, the chapter ends with a conclusion.

\section{Complexity Analysis}
  The complexities of all four programs are summarised in table
  \ref{table:allcomps}. Note that in most image processing applications $gmax < n$ holds. The number
  of pixels in an image is usually greater than the 
  number of gray tones each pixel has available.
  
  \begin{table}[h!]
    \caption{Work and Depth complexities}
    \label{table:allcomps}
    \centering
    \begin{tabular}{lll}
      \toprule
      program & work & depth \\
      \midrule
      \seq  & $n \log gmax + gmax$ & $n \log gmax + gmax$ \\
      \man  & $n \cdot gmax$ & $\log n$ \\
      \ndpn & $n \log n + gmax$ & $\log n + \log gmax$ \\
      \ndpv & $n \log n + gmax$ & $\log n + \log gmax$ \\
    \end{tabular}
  \end{table}
  A few observations can be made.
  
  \paragraph{\ndpn versus \ndpv:}
    \ndpn and \ndpv have equal work and depth complexities.
    Compiler optimisations can rarely optimise into a lower complexity class.
    However, the use of flat data structures, the optimisation
    for cache locality and the fusion of communication and loops
    greatly reduces the constants factors of \ndpv.
  
  \paragraph{\man versus \ndpv:}
    \label{paragraph:manvsndpv}
    Work and depth complexities of \man and \ndpv are competing.
    While the work of \man is a product of the parameters $n$ and $gmax$ - 
    the work of \ndpv (and \ndpn) is only a sum. For larger parameters,
    and a limited number of processors \ndpv will out-compete \man.
    \ndpv has a speed-up linear in $gmax$.
    
    Regarding depth however, the opposite is the case. \man grows
    logarithmic to $n$ whereas \ndpv grows logarithmic
    to $n$ and $gmax$. For large values of $gmax$ and a high number of processors,
    \man will out-compete \ndpv. However, the speed-up is only a
    summand of $\log gmax$.
  
  \paragraph{Sequential versus Parallel:}
    \seq has the best bounds in work out of all programs.
    It is low due to its iteration-based histogram creation.
    Parallel programs cannot use this method because they
    then were to fall back to sequential traversal.
    \man and \ndpn instead used more advanced methods to implement the histogram creation
    in parallel.
    
    On the one hand, their methods have worse work complexities.
    If $gmax$ is treated as a constant, then \seq grows linearly
    - whereas \man and \ndpv grow $O(n \log n)$.
    On the other hand, they greatly improve in depth complexities.
    While \seq remains linear to $n$, \man and \ndpv are only logarithmic to $n$.
    
    Parallel Programs have an overhead when the number of
    PUs is low. With an increasing number of PUs, they out-compete \seq.
  
  \paragraph{Parameter configuration:}
    Different parametrization in $n$ and $gmax$ leads
    to different programs being better and worse and others.
    As stated in paragraph \ref{paragraph:manvsndpv}, for
    large parameters and a limited number of PUs,
    \ndpv performs better than \man.
    However, for large $gmax$ and a high number of PUs the situation
    is reversed.
    Depending on the application context, different parameters
    $gmax$ and $n$ are given. This makes some algorithms faster than the other.
    
  
  \section{Pro and Contra}
    Table \ref{table:procons} gives an extensive
    pro and contra analysis of the four programs.
    They compare the programs on similarity to \ac,
    efficiency with many or few PUs,
    the human workload involved and more.
    
    \begin{table}[h!]
      \caption{Pros and Contras of the programs}
      \label{table:procons}
      \begin{center}
      \begin{tabular}{ll}
          \toprule
          program & pro \\
          \midrule
          \seq & \pro It is a direct implementation \ac \\
           & \pro It requires the least amount of human work\\
           & \con It is purely sequential and cannot gain from parallelism. \\
          \midrule
          \man & \pro It achieves time logarithmic to $n$ for large number of PUs \\
           & \pro It has low constant factors. \\
           & \pro It has a short implementation. \\
           & \blt{+/-} The programmer trades flexibility and comfort for fine grained control. \\
           & \con The work is product of $n$ and $gmax$. It grows faster the other ones. \\
           & \con It is only a surface translation of \ac \\
           & \con Normalisation and scaling and histogram creation were fused together manually. \\
           & \ind They are not separated steps anymore. \\
           & \con Much human work was necessary to parallelize the histogram creation. \\
           & \con Subsequent algorithms have to be coded to operate on flat images. \footnotemark \\
          \midrule
          \ndpn & \pro It is almost a direct implementation of \ac \\
           & \pro It implements parallel histogram creation using only high level function compositions \\
           & \pro It involves only limited workload for the human. \\
           & \con It has higher constant factors due to communication \\
          \ndpv & \\
           & \pro The programmer does not need to care about flat representation of the image \\
           & \pro Flattening of arrays and nested functions reduce constant factors \\
           & \pro Communication Fusion and Stream Fusion reduce constant factors\\
           & \ind and automatically fuse normalisation and scaling. \\
           & \pro The compiler optimises automatically \\
           & \con If failed, compiler optimisations are difficult to guide. \\
           & \blt{+/-} The programmer trades fine grained control for flexibility and comfort. \\
      \end{tabular}
      \end{center}
    \end{table}
    
  \section{Conclusion}
    TODO.
  
  \footnotetext{Unless one wraps \man with (un-)flattening operations. This approach however manually implements the flattening approach used in NDP.}

