

\epigraph{\emph{
%"I never think of the future. It comes soon enough."
"I claim to be a simple individual \\
liable to err like any other fellow mortal. \\
I own, however, that I have humility \\
enough to confess my errors \\
and to retrace my steps.
}}{
%Albert Einstein
Mahatma Gandhi
}

\section{Summary}
  This thesis gave an evaluation of Nested Data Parallelism in
  the functional programming language Haskell by giving
  four implementations of \algo and comparing them to each other.
  They were \seq, \man, \ndpn and \ndpv. The results suggest
  that no program was clearly better than all
  other programs. Each of them had its advantages and disadvantages.
  While \seq performs the best on a few number of processors,
  it is out-competed by parallelized programs when the number rises.
  \man has the lowest complexity for large number
  of processors - though it has the highest complexity for small numbers.
  However, \ndpv is a good compromise between both extremes.
  It enables almost as much abstraction as does \seq
  while still performing well on varying number of processors.
  
  This leads to the conclusion, that Nested Data Parallelism is a
  very welcome programming model in functional programming.
  

\section{Related Work}
  There is much active research in parallel programming
  in general and in the programming language Haskell.
  
  Generally, there are two approaches on parallel programming.
  In the following a summary of these approaches is given. The
  comments after the semi-cola state properties of problems
  for which the approach is fruitful.
  \begin{itemize}
    \item Task Parallelism; irregular, non-deterministic
    \begin{itemize}
      \item dependence graph-based: \cite{GenTaskPar2015}, deterministic, medium-abstraction
      \item futures-based: \cite{FutPar2012}, simple integration, subtle-preconditions 
      \item thread-based: \cite{POXISthreads1997}, high-control, low-abstraction
      \item In Haskell: \cite{Marlow2012Parallel}
      \begin{itemize}
        \item Algorithm + Strategy = Parallelism: \cite{Trinder1998Algorithm}; task-based, deterministic, subtle-preconditions
        \item The Par Monad; \cite{Marlow2011Monad}; dependency graph, in-concise, high-performing
      \end{itemize}
    \end{itemize}
    \item Data Parallelism: \cite{DataParallel1986}; (ir-)regular, deterministic
    \begin{itemize}
      \item Flat GPU Parallelism: CUDA \cite{CUDA2008}, OpenCL \cite{OpenCL2010}; highly regular and high-performing, deterministic, 
      \item In Haskell: \cite{Marlow2012Parallel}
      \begin{itemize}
          \item Repa: \cite{Keller2010Regular}; nested arrays, regular, data-parallel, high-performing
          \item Accelerate: \cite{McDonell2013Optimising}; GPU, high-performing, data-parallel, regular
          \item Nested Data Parallel: \cite{Harness2008}; type-dependent, irregular/flexible, concise
      \end{itemize}
    \end{itemize}
  \end{itemize}
  All related work on parallelism in Haskell (except for \cite{Marlow2011Monad})
  implement sophisticated optimisations strategies like NDP. In
  functional programming, these optimisations are more necessary than an option.
  Unoptimised functional programs are usually slower by a magnitude than their imperative counterparts.
  
  
\section{Future Work}
  Based on the results on this thesis, further work can be of interest:
  
  \subsection*{Brent's Equation}
    The work and depth complexities calculated in this thesis are mere complexitiy classes.
    Given more time and details on the implementation of the NDP functions,
    one could give an exact expression for the work and depth of each program.
    One could then use Brent's Equation \cite{Brent2011}:
    
    $$ \frac{W}{P} \leq T \leq \frac{W}{P} + L \cdot D $$
    
    where $P$ is the number of processors,
    $L$ is the communication latency in number of machine instructions,
    and $W$ and $D$ are work and depth in number of instructions, respectively.
    
    Brent's Equation gives upper and lower bounds on
    the time a program with work $W$ and depth $D$
    requires given a latency of $L$ and $P$ processors.
    
    If given exact numbers for work and depth (of table \ref{table:allcomps}), one could do a
    more fruitful and exact analysis in the parameter-space
    spanned by $n$, $gmax$, $P$ and $L$.
    For concrete parameters one could solve for the optimal
    algorithm to use.
  
  \subsection*{Alternate Algorithms}
    Given the flexibility of NDP, alternate algorithms and data structures can be considered.
    The original papers \cite{Harness2008} use the 'Banes-Hut-Algorithm' for
    their examples. It is an $O(n \log n)$ algorithm for the approximate of
    calculation of n-bodies moving in three-dimensional space under the
    force of gravity. A comparison with manual implementation is
    a relevant question - since the algorithm is known
    to be hard to parallelize. For image processing,
    one could similarly consider other algorithms such as
    'Connected-Components-Labelling' or 'Face-Detection'.
    
  \subsection*{Distributed NDP}
    The implementation of NDP in Haskell is much more general than its original
    implementation in NESL. Considering the explicit distinction of locality (e.g the use of \c{Dist})
    used, there might be only limited work necessary to expand NDP from working on a single machine
    with many processors to working on multiple machines with multiple processors.

\comment{

\section{Retrospectare Persona}
  I am impressed of myself on how much work was involved in this thesis -
  given the time-frame of mere three months. I am satisfied with this
  thesis - though I went to different paths on many directions. I
  had great fun in dealing with a few topics of personal interest
  and lifting it into an entire thesis. Even though it was fragile
  in the one of other aspect - it did however turn out
  well. Given my immediate future path, I am to say goodbye
  - and thank for the past three years at the HAW Hamburg.
  I especially want to thank my parents for their continuous support
  and my friends for ...

}

\section{Final words}
  With this paragraph, the thesis comes to an end.
  The thesis gave a short evaluation of Nested Data Parallelism
  in Haskell for implementations of \algo.
  The author thanks greatfully for your attention.
