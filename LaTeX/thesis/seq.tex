

This chapter introduces \seq - a sequential implementation
of \algo. The implementation is given first. Then
its work and depth complexities - as introduced in section \ref{section:parmeasures}
- are given.


\section{Implementation}
  \seq is a direct implemenation of the description of \algo
  in section \ref{section:hbalanceintro}.
  First the data structures used are presented.
  Then each component of the implementation is given. Afterwards
  the components are assembled to \seq.
  
  \paragraph{Data Structures}
    The implementation uses two data types.
    \begin{lstlisting}
type Image  = PtrVector (PtrVector Int)
type Hist a = TreeMap Int a
    \end{lstlisting}
    \c{PtrVector a} is a pointer-based array holding values
    of type \c{a}. The use of pointers enables them to be nested.
    Thus they can be directly used to represent two-dimensional images.
    \c{TreeMap k a} is a binary search tree indexed by keys of type
    \c{k} and containing values of type \c{a}. They are
    used for the representation of a histogram.
    \footnote{The use of arrays for the representation of
    histograms has undesireable effects. Languges with
    mutability can destructively update the array
    for each pixel in the image. The Immutability
    of Haskell prevents this and forces the use of
    immutable and shared data-structures like \c{TreeMap k a}.}
    
    Functions over \c{PtrVector} are suffixed \c{-V}.
    Functions over \c{TreeMap} aresuffixed \c{-M}.
  
  \paragraph{Histogram Calculation}
    The steps for the creation of the initial histogram are given below:
    \begin{lstlisting}
hist :: Image -> Hist Int
hist = foldrV (\i -> insertWithM (+) i 1) emptyM . concatV
    \end{lstlisting}
    \c{hist} proceeds in two steps. First the image is flatten
    into an one-dimensional array. Then, a \c{TreeMap} is created
    counting the number of occurrences of each gray tone.
    \c{foldrV} is linear in the size of the flattened iamge array.
    \c{insertWithM} is logarithmic in the number elements
    inserted into the map. The size is bounded by the number of 
    gray tones - namely \c{gmax+1}.
    
  \paragraph{Accumulation}
  Calculation the accumulated histogram can be implented by
  a prefix sum over values. It is linear in the size of the map.
  \begin{lstlisting}
accu :: Hist Int -> Hist Int
accu = scanlM (+) 0
  \end{lstlisting}
    
  \paragraph{Normalisation}
  After accumulation, one has to normalise the histogram.
  The normalisation is a direct implementation of its formula in
  \ref{section:hbalanceintro}.
  \begin{lstlisting}
normalize :: Int -> Int -> Hist Int -> Hist Double
normalize a0' agmax' as =
    let a0 = fromIntegral a0'
        agmax = fromIntegral agmax'
        divisor = agmax - a0
    in  mapM (\freq' -> (fromIntegral freq' - a0) / divisor) as
  \end{lstlisting}
  It applies the mapping over the tree-mapusing \c{mapM}.
  \c{fromIntegral} explicitly converts from \c{Int} to \c{Double} since
  Haskell clearly distinguishes them. Variable names with a prime
  (') denote values of type \c{Int}. Variable names without a prime
  denote \c{Double}s. This naming convention is equally used in
  \man, \ndpn and \ndpv.
    
  \paragraph{Scaling}
  Scaling occurs similar to normalisation. It is implemented
  by a mapping over all values in the histogram.
  \begin{lstlisting}
scale :: Int -> Hist Double -> Hist Int
scale gmax = mapM (\d -> floor (d * fromIntegral gmax))
  \end{lstlisting}
    
  \paragraph{Apply}
  The application of the gray tone mapping to the images pixels
  is implemented by a nested \c{mapV} over the image. It uses
  \c{lookupLessEqualM} to lookup the values for the histogram.
  (It reverts back to a lower gray tone, if the gray tone is not found in the map.)
  \begin{lstlisting}
apply :: Hist Int -> Image -> Image
apply as img = mapV (mapV (lookupLessEqualM as)) img
  \end{lstlisting}
  
  \paragraph{\algo}
  Having defined the components, one can now directly define \seq:
  \begin{lstlisting}
hbalance :: Image -> Image
hbalance img =
  let h = hist img
      a = accu h
      a0 = firstM a
      agmax = lastM a
      n = normalize a0 agmax a
      gs = scale gmax n
      img' = apply gs img
  in img'
  \end{lstlisting}
  First the histogram is created (line 3). Then it is accumulated (line 4).
  After that it is normalised (line 5 to 7) and scaled(line 8).
  And finally, the gray tone mapping is applied and returned (line 9 to 10).
  It is defined exactly as previosuly envisioned.
  
\section{Complexities}
  % TODO: continue here.
  USE COST-CENTRE PDF!
  
  Do runtime analysis like in analysis/seq.tex. But a bit less detailed.
  (The actual derivations could be packed into an appendix!)
  
  
  Introduce variables.
  \begin{itemize}
    \item n sei die Anzahl der Bildpixel
    \item w sei die Bildbreite
    \item h sei die Bildhöhe
    \item p sei die Anzahl der PUs (gang members).
  \end{itemize}
