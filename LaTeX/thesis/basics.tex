
% Zitat: Auf den Schultern von Riesen.

This chapter will give an introduction into the basics needed to
follow this thesis. We will be introduced into functional programming
in Haskell to ease the understanding of the presented code.
Then we will cover Nested Data Parallelism (NDP) where
key insights - and concepts to make use of them - will be presented.
Afterwards a short introduction into parallel complexity measures is given.
We will learn about work and depth complexities, how they relate
to the runtime duraiton and how they can be calculated.
Finally, the problem of \algo itself is presented - along with
a description on how it works and examples for its uses in image processing.

\section{Haskell}
  introduction into syntax, immutability, and higher-order functions, lambdas, currying
  polymorphic types (PA a, Dist a, Vector a)
  evaluation, record syntax, naming conventions functions f,g,h, vars x,y,z,a,b,c, arrays of x -> xs -> xss
  identity function, flip
\section{Nested Data Parallelism}
  Contrast Nested vs Flat data parallelism,
  Nested Data Parallelism in Haskell, an ongoing project.
  This work uses many ideas from their work.
  Show functions used to program in NDPH.
  Show vectorized types and data for AInt, ATup2, AArr and AClo/Clo.
  Transformation and explain main types (#25 in Notizblock).
  "Explicit calls to AArr and ATup2 and such mean global communication"
  "Types of PAs are global. Types of Dist are local."
    
  % TODO: explain with image mapP.png !!
  Each occurrence of \c{mapP f} is replaced by \c{fL} where \c{fL} is the lifted version
  of that function. For user defined-functions, this lifted function
  is automatically defined using other parallel primitives. However, for
  some special functions, there exists a special lifted implementation.
  We are concerned about nested parallel constructs like \c{mapP (mapP f)}.
  They are transformed into the following construct
  \begin{lstlisting}
  mapP (mapP f) xss = unconcatPS xss
                      . fL (expandPS xss fEnv)
                      . concatPS
                      $ xss
  \end{lstlisting}
  This is the key insight in nested data parallelism! Nested parallel operations
  can be reduced to flattening the array (line 3), applying a flat data-parallel operation (line 2)
  and finally unflattening the new array into the original structure (line 1).
  
  Show that GHC supports Inlining and Fusion through Rewrite Rules.
\section{Parallel Computing and Complexity Measures}
  Explain Work and Depth Complexity measures
  and their definitions!
   sumP implementierung als Beispiel für Laufzeiteinschätung mit Work/Depth und Anzahl von Prozessoren #1!
   
   (und zeigen, dass naives stream fusioning die parallelität verliert)
  
  $ cite: Parallel Programming Algorithms, NESL, Guy Belloch.
   
\section{\algo}

  Introduction into Histogram Balancing.
  
  Prefix sum is a very common operation in computer science. It is a special
    case of scanning through a ordered container from left ro right applying a binary
    associative function \c{f}.
    It is defined as:
      $$ scanl(f,z,[a_1,a_2,...,a_n])
         := [f(z,a_1),f(f(z,a_1),a_2),...,f(f(...f(f(z,a_1),a_2)...,a_{n-1}),a_n)]
      $$
    An example shall be $scanl(+,0,[1,2,2,3,-2]) = [1,3,5,8,6]$. In some definitions
    the first element is $z$. This is however not the definition we need here.
    
  
  Uses thereof and overview of an algorithm.
  Wird z.B im Zusammenhang mit \mu-Momenten zur Erkennung
    affin-transformierter Bildpaare verwendet.
