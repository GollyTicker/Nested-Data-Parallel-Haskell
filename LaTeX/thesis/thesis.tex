\documentclass[draft=false
              ,paper=a4
              ,twoside=false
              ,fontsize=11pt
              ,headsepline
              ,BCOR10mm
              ,DIV11
              ]{scrbook}
\usepackage[ngerman,english]{babel}
%% see http://www.tex.ac.uk/cgi-bin/texfaq2html?label=uselmfonts
\usepackage[T1]{fontenc}
\usepackage[utf8]{inputenc}     % Zum verwenden von ä,ü und ö: http://en.wikibooks.org/wiki/LaTeX/Special_Characters
\usepackage{libertine,pifont,microtype,textcomp,setspace,makeidx,listings,natbib,soul,hawstyle}
\usepackage[refpage]{nomencl}
\usepackage[colorlinks=true]{hyperref}
\usepackage{lipsum} %% for sample text
\usepackage[nottoc,notlot,notlof,numbib]{tocbibind}
\usepackage{xspace}

% My Commands
\newcommand{\comment}[1]{}

\newcommand{\seq}[0]{$P_{seq}$\xspace}
\newcommand{\man}[0]{$P_{man}$\xspace}
\newcommand{\ndpn}[0]{$P_{nest}$\xspace}
\newcommand{\ndpv}[0]{$P_{vect}$\xspace}
\newcommand{\algo}[0]{Histogram Balancing\xspace}
\newcommand{\pav}[1][a]{\c{PA #1}\xspace}
\newcommand{\pan}[1][a]{\c{[:#1:]}\xspace}
\newcommand{\pad}[1][a]{\c{Dist (PA #1)}\xspace}
\newcommand{\type}[1]{\c{#1}}
\newcommand{\W}[0]{\textrm{W}}
\newcommand{\D}[0]{\textrm{D}}
\newcommand{\p}[0]{\paragraph{}\xspace}
\renewcommand{\c}[1]{\mbox{\texttt{\footnotesize{#1}}\xspace}}
\newcommand{\lam}[0]{\textrm{$\lambda$}}

% Building:
% Gedit -> LaTeX build - Strg + Alt + 1
% $ bibtex thesis
% $ makeindex thesis.nlo -s nomencl.ist -o thesis.nls
% Gedit -> LaTeX build - Strg + Alt + 1

% My Packages
\usepackage{booktabs,url}
\usepackage[normalem]{ulem}
\usepackage{graphicx, amsmath}

%% define some colors
\definecolor{keywordblue}{HTML}{5F49CC}
\colorlet{BackgroundColor}{gray!10}
\colorlet{KeywordColor}{keywordblue}
\colorlet{CommentColor}{black!90}
%% for tables
\colorlet{HeadColor}{gray!60}
\colorlet{Color1}{blue!10}
\colorlet{Color2}{white}
  
%% configure colors
\HAWifprinter{
  \colorlet{BackgroundColor}{gray!20}
  \colorlet{KeywordColor}{black}
  \colorlet{CommentColor}{gray}
  % for tables
  \colorlet{HeadColor}{gray!60}
  \colorlet{Color1}{gray!40}
  \colorlet{Color2}{white}
}{}

\lstset{%
  numbers=left,
  numberstyle=\tiny,
  stepnumber=1,
  numbersep=5pt,
  %basicstyle=\ttfamily\small,
  keywordstyle=\color{KeywordColor}\bfseries,
  identifierstyle=\color{black},
  %commentstyle=\color{CommentColor},
  backgroundcolor=\color{BackgroundColor},
  captionpos=b,
  fontadjust=true,
  % mine:
  % frame=none,
  xleftmargin=1em,
  belowcaptionskip=\bigskipamount,
  captionpos=b,
  escapeinside={*'}{'*},
  language=haskell,
  tabsize=0,
  emphstyle={\bf},
  commentstyle=\it,
  stringstyle=\mdseries\rmfamily,
  showspaces=false,
  columns=flexible,
  showstringspaces=false,
  morecomment=[l]\%,
  basicstyle=\ttfamily\footnotesize,
  % keywordstyle=\color{keywordblue},
  keywords={type,if,then,else,let,in}
}
\lstset{escapeinside={(*@}{@*)}, % used to enter latex code inside listings
        morekeywords={uint32_t, int32_t}
}
\ifpdfoutput{
  \hypersetup{bookmarksopen=false,bookmarksnumbered,linktocpage}
}{}

%% more fancy C++
\DeclareRobustCommand{\cxx}{C\raisebox{0.25ex}{{\scriptsize +\kern-0.25ex +}}}

\clubpenalty=10000
\widowpenalty=10000
\displaywidowpenalty=10000

% unknown hyphenations
\hyphenation{
}

%% recalculate text area
\typearea[current]{last}

\makeindex
\makenomenclature

\begin{document}

% \selectlanguage{ngerman}

%%%%%
%% customize (see readme.pdf for supported values)
\HAWThesisProperties{Author={Chandrakant Swaneet Kumar Sahoo}
                    ,Title={v0.8 || Nested Data Parallelism for Image Processing Algorithms}
                    ,SubTitle={Optimisations in the Functional Programming Language Haskell}
                    ,EnglishTitle={Nested Data Parallelism for Image Processing Algorithms}
                    ,ThesisType={Bachelorarbeit}
                    ,SubTitleDelimiter={\, -}
                    ,ExaminationType={Bachelorprüfung}
                    ,DegreeProgramme={Bachelor of Science Angewandte Informatik}
                    ,ThesisExperts={Prof. Dr. Michael Köhler-Bußmeier \and Prof. Dr.-Ing. Andreas Meisel}
                    ,ReleaseDate={23. Juli 2015}
                  }

%% title
\frontmatter

%% output title page
\maketitle

\onehalfspacing

%% add abstract pages
%% note: this is one command on multiple lines
\HAWAbstractPage
%% German abstract
{Schlüsselwort 1, Schlüsselwort 2}%
{TODO Deutscher Abstract \ldots}
%% English abstract
{keyword 1, keyword 2}%
{TODO English Abstract \ldots}

\newpage
\singlespacing

\tableofcontents
\newpage
%% enable if these lists should be shown on their own page
\listoftables
%\listoffigures
% \lstlistoflistings

%% main
\mainmatter
\onehalfspacing
%% write to the log/stdout
%\typeout{===== File: chapter1}
%% include chapter file (chapter1.tex)

\chapter{Introduction (=2)}
  

\section{Context}
  In the past few decades the raw processing power of CPUs has
  not increased by any significant amount.
  Instead, we are presented with hardware with an
  exponentially increasing number of processors
  - each of them as fast as the previous generation.
  Parallel COmputing is concerned with the use of multiple
  processing units to increase performance.
  Functional programming languages are
  particularly successfull in parallel programming
  due to their immutability and cache friendly-ness
  
  Research and Development in this field is generally referred to as Parallel Computing.
  For my thesis, I have further reduced the context to the programming language Haskell and Image Processing. An explanation is given next.
    
  
  A few questions arise:
  
  !Context like in the expose!
  
  Why Image Processing? Simply interesting topic. Using Histogram Balancing
  
  
  Focus on practical functional programming.
    Why Haskell?
    many researchers do their work in the haskell community,
      on of them will be used here - that is Nested Data Parallel Haskell
    I enjoyed haskell the most and learned many concepts there. (like in expose)
    
    Results can (and are) applied to other functional programming
    lanuguages. cite? 
    
    
    Constant Factors

\section{Goal}

  To answer these questions, an image processing algorithm
  is chosen and implemented in four variations.
  
  \ac $:=$ \algo, the conceptual image processing algorithm
  
  \seq $:=$ An ordinary Haskell implementation of \ac

  \man $:=$ A manually-parallelized implementation of \ac
  
  \ndpn $:=$ A Nested-Data-Parallel implementation of \ac

  \ndpv $:=$ The compiler-transformed implementation of \ac from \ndpn
        

  Leading question. Subquestion.
  Formulating the approach
  
  Formulate Concept/Algorithm of Histogram Balanding as $A$.

\section{Structure}
  Describe what each chapter does



\chapter{Basics}
  
% Zitat: Auf den Schultern von Riesen.

This chapter will give an introduction into the basics needed to
follow this thesis. We will be introduced into functional programming
in Haskell to ease the understanding of the presented code.
Then we will cover Nested Data Parallelism (NDP) where
key insights - and concepts to make use of them - will be presented.
Afterwards a short introduction into parallel complexity measures is given.
We will learn about work and depth complexities, how they relate
to the runtime duraiton and how they can be calculated.
Finally, the problem of \algo itself is presented - along with
a description on how it works and examples for its uses in image processing.

\section{Haskell}
  introduction into syntax, immutability, and higher-order functions, lambdas, currying
  polymorphic types (PA a, Dist a, Vector a), type synonyms
  evaluation, record syntax, naming conventions functions f,g,h, vars x,y,z,a,b,c, arrays of x -> xs -> xss
  identity function, flip

\section{Nested Data Parallelism}  
  In FDP, parallel mapping primitives are provided to express parallelism.
  E.g. we are provided with a function \c{map f xs}
  \footnote{Its type is \type{(a -> b) -> Array a -> Array b}}
  which applies \c{f} on each of the elements in \c{xs} in parallel.
  This function has a farily intuitive parallel implementation -
  simply distribute the input array across the processing units (PUs),
  make each PU compute the mapping for its local chunk and finally
  join all elements together.
  However, the inner function \c{f} has to be sequential.
  Because it is sequentially applied by each PU, the PU itself cannot
  apply it in parallel again.
  This is in constract to NDP where \c{f} itself can also be a parallel operation!
  Hence the name - \textbf{Nested} Data Parallelism.
  
  Let's take a look at an example to show the difference between FDP and NDP.
  
  \subsection{SMVM}
    An widely known problem is sparse-matrix-vector-multiply (\c{smvm}).
    Given a sparse vector (an array with tuples of \c{(Idx,Elem)}) we want
    to multiply it with a sparse matrix
    \footnote{A sparse matrix is a matrix,
    where the elements of each row are represented
    by sparse vectors.}
    to obtain a result vector.
    Sprase matrices and sparse vectors are ver common
    in scientific applications and are used to represent
    vectors/matrices with only a few non-zero elements.
    
    In NDP we can give the following straightforward implementation:
    \begin{lstlisting}
type SparseVector = [: (Int,Double) :]
type SparseMatrix = [: SparseVector :]

dotp :: [:Double:] -> SparseVector -> Double
dotp v sv = sumP ( mapP ( \(i,x) -> (v!i)*x ) sv )

smvm :: SparseMatrix -> [:Double:] -> SparseVector
smvm sm v = mapP (dotp v) sm
    \end{lstlisting}
    We first define the dot product for a sparse vector with a dense vector.
    \c{dotp} works by assigning each element in the sparse vector,
    a new value by multiplying its old value with the corresponding
    value in the second vector
    \footnote{\c{v!i} accesses the element at the index i in the array v.}
    , and subsequently summing up all the elements
    \footnote{\c{sumP} in this context has the type \type{[:Double:] -> Double} and
    is implemented with a parallel tree-style logarithmic reduction.}
    .
    It can be executed optimally within FDP because the inner fuction is a sequential operation.
    
    In \c{smvm} we multiply each row with the input vector by using
    the previously defined \c{dotp}. \c{smvm} is different.
    We are using a parallel operation \c{dotp} inside another parallel operation - namely \c{mapP}.
    FDP cannot cope with such constructs and it would only parallise the
    outer most operation (that is \c{mapP}). Each dot product would
    be executed sequentially.
    
    In contrast, within NDP \c{smvm} both levels of parallelism
    would be executed in parallel.
    It does that by transforming the program we wrote into a
    functionally equivalent flat data parallel program. This transformation
    is called 'flattening' or 'vectoriation' and will be explained in the next sections.
    
  \subsection{Vectorization}
    
    Consider a slightly simpler situation.
      
    \begin{lstlisting}
let result = mapP (mapP (\x -> x+1)) [[1,2,3],[4,5],[],[6]]
    \end{lstlisting}
    It increases the elements of a nested array by one.
    If we want to execute both levels in parallel,
    then we would need to apply the increment function on
    every element regardless of the nesting structure.
    
    
    \begin{lstlisting}
      mapP (mapP f) xss = unconcatPS xss
                          . fL (expandPS xss fEnv)
                          . concatPS
                          $ xss
    \end{lstlisting}
    
    
    
    
  % TODO: was hiermit tun?
  In the ground breaking work \cite{Belloch1996}
  mayjor contributions to NDP were made. The paper
  also presented Belloch's earlier work (\cite{NepaBelloch1993}) on NESL - a programming
  language specifically designed for expressing parallelism
  in functional programming languages. Its ideas and insights were
  adapted to various languguages. One of these languages is Haskell.
  Multitude years of research was nesessary to generalize the
  advantages of the special purpose language NESL to a widely used
  general purpose language like Haskell. This project
  is called Nested Data Parallel Haskell and this section will give
  an overview of its key contributions.
  
  \paragraph{A word on accuracy}
    The project is - even after 15 years -
    still in \textit{work in progress}. Due to frequent changes,
    the papers often use conflicting notation and refer to
    different statuses of progress. Inconsistent literature and a project still in work makes it difficult
    to apply it in a thesis. It is not simple to use the original ideas from
    NESL directly on Haskell as there are great differences (not mentioning
    the fact it already took 15 years to adapt).
    
    Therefore, in this thesis, I have improvised on various conflicting or
    missing details. I assumed implementations which could really have been
    used in NDP.\footnote{E.g. \c{groupP} as introducted in chapter 5 doesn't
    even exist right now. Its implementation as described is however perfectly possible.}
    The reader is hereby noted that the details
    mentioned here are implementable - but not nesessarily an accurate
    representation of the current state of progress.
  
  
  
  % TODO: needs citations.

  Show functions used to program in NDPH.
  Show vectorized types and data for AInt, ATup2, AArr and AClo/Clo.
  Transformation and explain main types (#25 in Notizblock).
  "Explicit calls to AArr and ATup2 and such mean global communication"
  "Types of PAs are global. Types of Dist are local."
    
  % TODO: explain with image mapP.png !!
  Each occurrence of \c{mapP f} is replaced by \c{fL} where \c{fL} is the lifted version
  of that function. For user defined-functions, this lifted function
  is automatically defined using other parallel primitives. However, for
  some special functions, there exists a special lifted implementation.
  We are concerned about nested parallel constructs like \c{mapP (mapP f)}.
  They are transformed into the following construct
  This is the key insight in nested data parallelism! Nested parallel operations
  can be reduced to flattening the array (line 3), applying a flat data-parallel operation (line 2)
  and finally unflattening the new array into the original structure (line 1).
  
  
  Show that GHC supports Inlining and Fusion through Rewrite Rules.
\section{Parallel Computing and Complexity Measures}
  Explain Work and Depth Complexity measures
  and their definitions!
   sumP implementierung als Beispiel für Laufzeiteinschätung mit Work/Depth und Anzahl von Prozessoren #1!
   
   (und zeigen, dass naives stream fusioning die parallelität verliert)
  
  $ cite: Parallel Programming Algorithms, NESL, Guy Belloch.
   
\section{\algo}

  Introduction into Histogram Balancing.
  
  Prefix sum is a very common operation in computer science. It is a special
    case of scanning through a ordered container from left ro right applying a binary
    associative function \c{f}.
    It is defined as:
      $$ scanl(f,z,[a_1,a_2,...,a_n])
         := [f(z,a_1),f(f(z,a_1),a_2),...,f(f(...f(f(z,a_1),a_2)...,a_{n-1}),a_n)]
      $$
    An example shall be $scanl(+,0,[1,2,2,3,-2]) = [1,3,5,8,6]$. In some definitions
    the first element is $z$. This is however not the definition we need here.
    
  
  Uses thereof and overview of an algorithm.
  Wird z.B im Zusammenhang mit \mu-Momenten zur Erkennung
    affin-transformierter Bildpaare verwendet.

  
\chapter{Sequential: \seq}
  


  The sequenial algorithm is a straightforward implementation of the previosly
  described algorithm for \algo.
  
  \section{Implementation}
    Show code,
    Reexplain syntax,
    Explain how it works in words (uses binary-tree maps, dense-arrays,etc..)
    
  \section{Complexities}
    Do runtime analysis like in analysis/seq.tex. But a bit less detailed.
    (The actual derivations could be packed into an appendix!)
    
    
    Introduce variables.
    \begin{itemize}
      \item n sei die Anzahl der Bildpixel
      \item w sei die Bildbreite
      \item h sei die Bildhöhe
      \item p sei die Anzahl der PUs (gang members).
    \end{itemize}

  
\chapter{Manually parallised: \man}
  
\epigraph{\emph{
"When we had no computers, we had no programming problem either.
When we had a few computers, we had a mild programming problem.
Confronted with machines a million times as powerful,
we are faced with a gigantic programming problem. "
}}{
Dijkstra, 1998
}

This chapter will give an implementation of \algo
that cannot make use of NDP. This is
the case when one is only given a few parallel primitives
that don't support nesting
of operations (at least not without falling back to sequential evaluation).
Table \ref{table:parprims} shows a few of these primitives.

  \begin{table}[h!]
    \caption{Flat Data-Parallel Primitives}
    \label{table:parprims}
    \begin{center}
    \begin{tabular}{lll}
      \toprule
      function & type \\
      \midrule
      parMap & \c{(a -> b) -> Vector a -> Vector b} \\
      parZipWith & \c{(a -> b -> c)} \\
       & \c{-> Vector a -> Vector b -> Vector c} \\
      parReplicate & \c{Int -> a -> Vector a} \\
      parGenerate & \c{Int -> (Int -> a) -> Vector a} \\
    \end{tabular}
    \end{center}
  \end{table}
  They are analogous to the parallel functions in NDP.
  \c{parGenerate} is a function such that \c{parGenerate size f} creates a new array
  of size \c{size} and uses the generator function \c{f} to create
  the elements by their indices.
  E.g. \c{parGenerate 5 (\lam i -> i*i) = [0,1,4,9,16] }.
  These primitives all have work $O(n)$ and depth $O(1)$.
  
\section{Parallel histogram accumulation}
  To implement \algo in parallel, one has to revisit the sequential
  implementation. The parallel creation of the accumulated histogram is
  a difficult task. The goal is to try to come up with
  an low complexity algorithm for the histogram calculation.
  After a few tries, one can give the following implementation:
  
  \begin{lstlisting}
accuHist :: Image -> Hist
accuHist []  = parReplicate gmax 0
accuHist [x] = parGenerate gmax (\i -> if (i >= x) then 1 else 0)
accuHist xs  = let (left,right) = splitMid xs
                   [leftRes,rightRes] = parMap accuHist [left,right]
               in  parZipWith (+) leftRes rightRes
  \end{lstlisting}
  The general idea is to merge accumulation and histogram creation
  into a single tree-like reduction. To each array of pixels, \c{accuHist}
  returns the accumulated histogram of its gray tones.
  The algorithm can be broken down into two edge-cases and one recursive case
  . Figure \ref{figure:accuHist} gives an example of its evaluation.
  
  \begin{figure}[h]
    \centering
    \includegraphics[width=\linewidth]{accuHist}
    \label{figure:accuHist}
    \caption[Parallel Histogram Accumulation]{Evaluation of \c{accuHist [1,3,1,0]}}
  \end{figure}
  
  Suppose \c{gmax = 4} then the algorithm returns \c{[0 0 0 0 0]} if
  the input array was empty. If the image only contained a single gray tone
  \c{x}, then it creates its accumulated histogram
  \c{[0 0 ... 0 1 ... 1 1]} such that \c{x} is the index
  of the first \c{1}.
  For example, for \c{x = 2} the array
  \c{[0 0 1 1 1]} is returned. This is implemented using the \c{parGenerate} function.
  Finally, one has the recursive case. In this case, the the calculation
  of larger images is broken down by splitting the array into half
  \footnote{splitting is a constant time operation
  for these view-based arrays} and applying \c{accuHist}
  recursively. Finally, the histograms are merged with element-wise addition
  ({\c{zipWith(+)}}).
  
  This algorithm was carefully constructed after many failed approaches
  on such an algorithm. Alternatives were considered, but they did not
  yield an acceptable complexity compared to \c{accuHist}.

\section{Implementation}
  Given the introduction, the manually parallelized \man can finally be implemented.
  The parallel primitives can be integrated
  well into normalisation and scaling.
  However, \c{accuHist} and \c{apply} need to be adapted.
  The code is given below. \c{(!)} is used for indexing.
  \begin{lstlisting}
type Image  = Vector Int
type Hist   = Vector Int

hbalance :: Image -> Image
hbalance img =
  let as = accuHist img
      a0 = as ! 0
      agmax = as ! gmax
      
      sclNrm x = floor ( (x-a0)/(agmax - a0)*gmax )
      gs = parMap sclNrm as
      
      apply gs = parMap (\i -> gs ! i) img
      img' = apply gs img
      
  in  img'
  \end{lstlisting}
  As explained in the previous section, the parallel
  accumulated histogram calculation has been implemented in \c{accuHist}.
  For the gray tone mapping (\c{apply}) to work, nested arrays cannot be used
  \footnote{as they would result into an array of pointers to sub-arrays.
  This is undesirable due to Cache Locality.}
  .
  One needs to change the entire image representation to a flat array manually.
  To retrieve a specific pixel one needs to calculate
  the offset using the image's width. Fortunately,
  for \algo, indexed retrieval of pixels is not needed.
  However, any subsequent algorithms in the pipeline of image processing
  would have to cope with the flat image representation directly.
    
\section{Complexities}
  In this section, complexity measures for the functions
  involved in \man will be given.
  To calculate work and depth of \man, one needs the measures of
  all sub-functions and sub-expressions. \c{accuHist} is
  not a built-in function - and so needs an individual analysis first.
  
  \paragraph{\c{accuHist}}
    The recursive case of \c{accuHist} involves functions
    of work $O(gmax)$ and depth $O(1)$ - 
    namely \c{parZipWith (+)} and \c{splitMid}.
    The exceptions are the two recursive calls (packed together into a
    \c{parMap}).
    
    One can state the work of \c{accuHist} as a recursive function
    \begin{equation*}
    \W(n,gmax) = \begin{cases}
                 gmax & \text{ if } n \le 1 \\ 
                 2 \W(\frac{n}{2}) + gmax & \text{ else }
                \end{cases} \\
    \end{equation*}
    where the edge-cases and recursive-cases correspond one-to-one
    to the definition of \c{accuHist}.
    Such a recurrence relation can be resolved by tying the knot
    or using the Master Theorem's first case
    \footnote{Master theorem: \cite{Cormen2001}}
    \footnote{However, the Master Theorem does not apply directly
    because it treats \c{gmax} as a constant -
    and not as a variable parameter.
    The Master Theorem would give $O(n)$ whereas tying the knot
    would give the more accurate class $O(n \cdot gmax)$.}
    .
    The following equations shall tie the knot:
    \begin{equation*}
    \begin{split}
    \W(n,gmax)
      & = \begin{cases}
            gmax & \text{ if } n \le 1 \\ 
            gmax2^0 + 2^1\W(\frac{n}{2}) & \text{ if } n = 2 \\
            gmax2^0 + gmax2^1 + 2^2\W(\frac{n}{4}) & \text{ if } n = 3 \\
            gmax2^0 + gmax2^1 + ... + gmax 2^{\log n - 1} + 2^{\log n}\W(1) & \text{ else }
          \end{cases} \\
      & = \textrm{ (... tying the knot ...) } \\
      & = gmax \sum_{i=0}^{\log n}{2^i} \\
      & = gmax (2^{\log n + 1} - 1) \\
      & = gmax (2n - 1) \\
      & \in O(n \cdot gmax) \\
    \end{split}
    \end{equation*}
    
    The work involved is an product of the number of gray tones and
    the number of pixels $O(g \cdot gmax)$.
    Such a tree-like reduction has height logarithmic
    in the size of the input array. The input array is the image and
    has size \c{n}. Therefore, one can conclude
    $\D(n, gmax) \in O(\log n)$.
  
  \paragraph{Putting it together}
    Given the code for \man, one can now give work and depth
    complexities. These complexities are given in table \ref{table:man}.
    The table summarises the work and depths of each of the calls.
    It is calculated by applying the formulas from section
    \ref{section:parmeasures}.
    
  \begin{table}[h]
    \centering
    \caption{Complexities for \man}
    \label{table:man}
    \begin{tabular}{lll}
        \toprule
        function or variable & $\W \in O(...)$           & $\D \in O(...)$ \\
        \midrule
        hbalance        & n \cdot gmax    & $\log$ n \\
        apply           & n           & 1 \\
        parMap sclNrm   & gmax        & 1 \\
        accuHist        & n  \cdot gmax    & $\log$ n \\
        \midrule
        accuHist        & n  \cdot gmax    & $\log$ n \\
        splitMid        & 1           & 1 \\
        parZipWith      & gmax        & 1 \\
        parReplicate    & gmax        & 1 \\
        parGenerate     & gmax        & 1 \\
        arr ! i         & 1           & 1 \\
    \end{tabular}
  \end{table}
  
  
  The outer-most work and depth of \man is given below:
  \begin{equation*}
  \begin{split}
  \W(n,gmax)
        & = \W(accuHist) + \W(parMap,sclNrm) + \W(apply,gs) \\
        & = n \cdot gmax + gmax + n \\
        & \in O(n \cdot gmax) \\
  \D(n,gmax)
      & = \D(accuHist) + \D(parMap,sclNrm) +  \D(apply,gs) \\
      & = \max \{ \log n, 1, 1\} \\
      & \in O(\log n) \\
  \end{split}
  \end{equation*}
  
  One can note, how work and depth of \man is entirely bounded
  by the complexities of \c{accuHist}. Improvements to \c{accuHist}
  complexities will improve \man either.
  
  \paragraph{}
  Before moving to the next chapter - one shall be reminded that
  \man involved much manual work. It was not a direct translation
  of the algorithms description. It, especially requires
  the subsequent algorithms to use a flat image representation.
  
  \paragraph{}
  The next chapter covers \ndpn - an implementation using NDP.
  
  
  
  


\chapter{Nested-Data-Parallel: \ndpn}
  
  % Quine quotation.

  This chapter will describe an implementation of \algo which uses
  Nested Data Parallelism in Haskell (by \cite{Harness2008}).
  First, a a few predefined functions will be presented to
  increase their the understanding of their operational behavior.
  Then the implementation will be presented and it's complexities will
  calculated.
  
  \section{Approach}
  
    \subsection{Scanl}
      Parallel prefix sum has well studied efficient implementaitons. One of them
      is the following:
      \begin{lstlisting}   
scanlP f z xs =
  joinD
    . mapD (\(as,a) -> mapS (f a) as)
    . propagateD f z
    . mapD (scanlS f z)
    . splitD
    $ xs
      \end{lstlisting}
      This implementatin is designed to reduce communication
      and therefor increase efficiency. It works in three steps.
      First, each PU computes its local prefix sum (line 5).
      Second, the total sum of each of the PUs is propagated
      around adding up subsequent values.
      Third, the updated sum is used to increase the values of the local chunks.
      This approach is visualised in \ref{figure:scanlPsteps}
      
      \begin{figure}[h!]
          \includegraphics[width=\linewidth]{scanlP-three-steps.png}
          \caption{Parallel prefix sum in three steps (Figure from \cite{DistTypes1999}) }
          \label{figure:scanlPsteps}
      \end{figure}
      The propagation is the bottle neck in terms of dpeth and parallel complexity.
      But since the progapation itself is structurally isomorphic to prefix the entire sum,
      we can use a more efficient scheme for propagation - rather than using linear propagation.
      Propagation in logarithmic depth is possible and an be derived from \cite{Scanl1980}.
      For our purposes, it is sufficient to know the following complexities for scanlP:
      $\W(n) \in O(n)$ and $\D(n) \in O(\log n)$.

    \subsection{GroupP}
      \c{groupP} is a frequently used function in functional programming.
      Given an array it returns an array of arrays,
      where each subarray contains equal consecutive
      elements of the source array. For example
      \c{groupP [1,2,2,2,2,3,3,4]} becomes \c{[[1],[2,2,2,2],[3,3],[4]]}.
      In NDP, the latter is represented by
      \begin{lstlisting}
      AArr {
        data = [# 1,2,2,2,3,3,4 #],
        segd = ATup2 {
          as = [# 0,1,5,7 #]
          bs = [# 1,4,2,1 #]
        }
      }
      \end{lstlisting}
      The key insight in an efficient parallel implementation of \c{groupP}
      relies on the following insight - the data field in the nested array
      is the source array itself! To implement \c{groupP} we only
      need to efficiently calculate the segment descriptor field. This is
      possible in logarithmic depth of the size of the input array!
      
      \begin{figure}[h!]
          \includegraphics[width=\linewidth]{groupP.png}
          \caption{An example calculation of groupP. Each box is a PU.}
          \label{figure:groupP}
      \end{figure}
      
      As visualised in the figure \ref{figure:groupP}
      we can do that by spliting all elements onto all PUs first.
      Then each PU creates a local chunk of a linked list of
      \c{(Value,StartIdx,Count)}-Triplets to record it singleton.
      After that, each level of recursion merges two PUs by
      merging the last triplet of the left list with the first triplet
      of the right list. If both triplets correspond to the same value,
      then a new triplet with the total count and the left index is used.
      If both are unequal, then they are left unchanged. The implementation
      of \c{groupP} uses a subfunction \c{segdSplitMerge} to implement the
      splitting and merging. This function will be exposed lateron in
      the vectorization of \ndpn.
      Further analysis reveals the complexities $\W(n) \in O(n)$ and $\D(n) \in O(\log n)$.
      All in all, \c{groupP} is an operation which can very well exploit the flat representation of nested arrays.
      
    \subsection{SortP}
      General parallel sorting can be as simple as a parallel
      implementation of merge-sort where the recursive calls are executed
      in parallel. This comparision based sorting has complexities of
      $\W(n) \in O(n \log n)$ and $\D(n) \in O(\log n)$. 
      
  \section{Implementation}
    \begin{lstlisting}
type Image = [:[:Int:]:]
type Hist a = [:a:]

hbalance :: Image -> Image
hbalance img =
    let h = hist img
        a = accu h
        a0 = headP a
        agmax = lastP a
        n = normalize a0 agmax a
        s = scale gmax n
        img' = apply s img
    in  img'

hist :: Image -> Hist Int
hist = sparseToDenseP (gmax+1) 0
          . mapP (\g -> (headP g,lengthP g))
          . groupP
          . sortP
          . concatP

accu :: Hist Int -> Hist Int
accu = scanlP (+) 0

normalize :: Int -> Int -> Hist Int -> Hist Double
normalize a0' agmax' as =
    let a0 = fromIntegral a0'
        agmax = fromIntegral agmax'
        divisor = agmax - a0
    in  [: (fromIntegral freq' - a0) / divisor | freq' <- as :]

scale :: Int -> Hist Double -> Hist Int
scale gmax as = [: floor (a * fromIntegral gmax) |  a <- as :]

apply :: Hist Int -> Image -> Image
apply as img = mapP (mapP (as !:)) img
    \end{lstlisting}
  \section{Complexities}
    ...
    
    % TODO: Genau erläuteren und präzisieren wie Work&Depth mit der Anzahl der Prozessoren in den distributed Types zusammenhängen.
    %    Die tatsächliche Parallelität steck in der Anzahl der PUs (Processing Units) und den verteilten Algorithmen
    %    zwischen den einzelnen PUs. Damit wird sumD und propagateD auf D(log n) gedrückt.
        
    % Ignoriert!! Twofold interpretation: divL = <built-in parallel divL> OR < mapD divS> with distributed types and extended library optimization
    % immer die zweite variante - weil sonst das andere Pman program nicht ginge.
      
    


\chapter{Optimised Nested-Data-Parallel: \ndpv}
  \label{chapter:ndpv}
% TINA - There is no alterative.
%   - Angela Merkel

After building up momentum and viewing a few implementations of \algo
we are now ready to tackle the core transformation and optimisations
offered by Nested Data Parallelism in Haskell (and related functional languages).
This chapter will go through the transformations and present the
final program \ndpv. This program would have been the results of the compilers \textit{automatic}
optimisations before it would be translated into mashine code and finally executed.
At the end of this chapter, we will measure the complexities of \ndpv as we already did for the previous programs.

\section{Transformations}
  As presented in the overview, the compiler applies a series of transformations. They can be
  roughly broken down into three steps:
  \begin{enumerate}
    \item \emph{Vectorization} - Uses flat non-parametric array \pav representations instead of the
          vanilla nested parallel arrays \pan . The flatening is also applied to nested parallel comprehensions.
    \item \emph{Communicaiton Fusioning} - The parallel functions are inlined and the compiler uses rewrite rules to
          find and reduce communication inbetween PUs. It distinguishes between global arrays (\pav) and the local chunks (\pad)
          compromising the global one.
    \item \emph{Stream Fusioning} - Local sequential functions over the chunks can be further fused together. Now
          the creation of intermediate local arrays (\pad) is eliminated.
  \end{enumerate}
  
  Most notably, the vectorization step makes the following mapping.
  \begin{lstlisting}
mapP (mapP f) xss => unconcatPS xss
                          . fL
                          . concatPS
                          $ xss
  \end{lstlisting} % removed (expandPS xss fEnv)
  where \c{fL} is the lifted version of the original \c{f}.
  % TODO: words
  we can now finally take a look at the first step of the compilers transformation.
  
  \subsection{Vectorization}
    Applying the vectorization procedure as described in \cite{Harness2008} yields the following code. Let's take a look  at it.
    
    % After Vectorization:
    \begin{lstlisting}
hbalance img :: PA (PA Int)
hbalance img = 
let a = scanlPS plusInt 0
            . sparseToDensePS (plusInt gmax 1) 0
            . (\g -> ATup2 (headPL g) (lengthPL g))
            . groupPS
            . sortPS
            . concatPS
            $ img
    n = lengthPS a
    gs = floorDoubleL
           . multDoubleL (int2DoubleL (replPS n gmax))
           . divL
               (minusL
                (int2DoubleL a)
                (replPS n (int2Double (headPS a)))
               )
           . replPS n
           $ minusDouble (int2Double (lastPS a)) (int2Double (headPS a))
in unconcatPS img
     . indexPL
     . concatPS
     $ img
    \end{lstlisting} % (expandPS img gs)
    We can observe how our functions \c{hist},\c{accu} etc. have been inlined and are tightly packed together here.
    We have also moved from using nested \pan to flat \pav and we have replaced polymorphic funcitons like \c{fromIntegral}
    with specific monomorphic primitive mashine functions like \c{intToDouble}.
    
    Starting, lines 9 to 4 describe the calculation of the histogram.
    It's only difference is the use of vectorized scalar functions (e.g.\c{groupPS}). These functions operate of the efficient flat
    representation instead of the nested representation.
    
    
    After that in line 4, the accumulated histogram is calculated. Lines 19 to 13 and 21 to 11
    describe the normalisation and scaling of the gray tones respectively. The vectorized code uses
    lifted arithmetic functions (like \c{floorDoubleL}) which operate over arrays of values - in contrast to
    the scalar mashine primitives like \c{floorDouble}. Essentially, our variables \c{gmax'},\c{a0} and \c{divisor}
    have been inlined are now replicated\footnote{\c{replPS n x} creates an array of length n - all containing the element x. It has the type \type{Int -> a -> PA a}}
    to the length of the gray tone array before the lifted arithmetic operations are applied.
    
    
    Finally, lines 32 to 20 describe the mapping of the images gray tones.
    The nested parallel operation \c{mapP (mapP (!a))} - formerly a part of \c{apply} - 
    has now been flatten to use a lifted parallel operation, namely \c{indexPL}, over a \textbf{flat} array of
    the pixels of the image. This is the core of nested data parallelism!
    
    In total, the program runtime has a few smaller constants factors. This is mainly due to the elimination of nested data structures
    and operations. However, there is still much room for improvement.
    
    An undesireable side-effect
    is the replication of the variables during normalisation and scaling. Instead of creating three arrays of identical values
    to bulk-operate over them with the histogram array - we would like to create a single function which first
    captures them as in closure. We would then map over the histogram without creating the intermediate arrays.
    Luckily, our transformation is not over yet - and we will observe how this problem will be fixed and how further optimisations will be applied.
    Let's head over to the next step.
    
  \subsection{Communication Fusioning}
    
    Communication Fusioning consists of inlining definitions of parallel functions and using rewrite rules to eliminate
    unesessary communication. Applying the transformation resulted in a changed definition of the histogram \c{a} and the
    gray tones \c{gs}. Let's take a look at their new forms.
    
    \subsubsection{Histogram calculation}
      The new form of the accumulcated histogram calculation is given below:
      \begin{lstlisting}
let a = joinD                           -- scanlPS ends
          . mapD (\(as,a) -> mapS (plusInt a) as)
          . propagateD plusInt 0
          . mapD (scanlS plusInt 0)       -- scanlPS begins; fused
          . sparseToDenseD (plusInt gmax 1) 0 -- sparseToDensePS ends
          . splitSparseD (plusInt gmax 1)     
          . joinD                             -- sparseTodensePS begins
          . mapD tripletToATup2               -- fused lambda and groupPS 
          . segdSplitMerge 0                  -- workhorse of groupPS
          . sortPS
          . concatPS
          $ img
      \end{lstlisting}
      We can firstly observe the occurrence of functions with a \c{-D} suffix. They operate either
      on each PU locally (as with \c{mapD}) or implement some specific inter-PU calculation (as does \c{propagateD}).
      These functions are the result of inlining various parallel functions and eliminating communication.
      The correspondence to their original functions are given as comments in the code. Only sortPS and concatPS
      are unchanged. A real compiler would have inlined their definitions and looked for optimisations.
      We are ignoring their inlining here to simplify the code - there is not much room for optimisation underneath them.
      
      
      Aside from them, inlining \c{scanlPS} and \c{sparseToDensePS}
      \footnote{It's definition is given in the appendix.} releaves their internals.
      Inbetween both of the functions, there was a composition of the distribution primitives - namely \c{splitD . joinD}.
      Applying the rewrite rule "splitD/joinD" eliminated this communiation overhead. Hurray! Our compiler
      is now finally starting to detect and fuse nunessesary communication!
      
      This leaves \c{propagateD} as the only inter-PU communication inbetween the spliting of the
      sparse array and joining the histogram at the end (line 1).
      
      The lambda and the groupPS were involved in a rather special fusion. We won't
      go into the detail of it. Essentially, the lambda expression was
      was applied to the result of groupPS. This enabled further communication
      fusioning and created the local operation \c{tripletToATup2}. It creates the
      local chunks of the sparse-array directly. \c{segdSplitMerge} does the
      actual work of the distributed grouping\footnote{as explained in the previous chapter}.
      
    \subsubsection{Normalisation and Scaling}
    
    Now let's take a look at the remaining part of the code - that is - the
    normalisation and scaling procedure:
    
    \begin{lstlisting}
let n = lengthPS a
  gs = joinD . mapD f . splitD $ a
  f = (\gmax' divisor a0 x ->
        floorDoubleS
          (multDoubleS
            (divDoubleS
              (minusDoubleS
                (int2DoubleS x)
                a0)
              divisor)
            gmax')
        )
        $ ( replD n . int2Double $ gmax )
        $ ( replD n
            . minusDouble (int2Double (lastPS a))
            . int2Double . headPS $ a )
        $ ( replD n . int2Double . headPS $ a )
    \end{lstlisting}
    Let's take a look of the original definitions of the lifted functions. Their definition goes similar to these two examples:
    \begin{lstlisting}
floorDoubleL = joinD . mapD floorDoubleS . splitD
multDoubleL as = joinD . zipWithD multDoubleS (splitD as) . splitD
    \end{lstlisting}
    The lifted functions first split their argument arrays(\c{splitD}) into each PU.
    Then each PU (\c{mapD}) applies the respective operation on it's local chunk(\c{floorDoubleS}).
    Finally the local chunks are joined into a global array.
    
    Inlining these functions created \emph{five} pairs of \c{splitD . joinD} - which were then immediately
    eliminated using the "splitD/joinD" rule.
    
    After that, a cascade of rewrite rules fire and propagate the normalisation and scaling
    constants towards inside the lifting operations. A sophisticated combination of the rules
    "mapD/zipWithD", "splitD,replPS", "mapD/replD" and "ZipReplSplit" results in
    the code we currently have.
    
    Operationally, there an important change in the normalisation and scaling.
    Although, constants (like \c{int2Double \$ gmax}) are still being replicated into arrays
    before applying the arithmetic mapping - now the replication
    is only limited to the local PU. This is different than before, when
    the constants were replicated globally and subsequently split.
    We can observe, how the code is slowly nearing our previously intuitioned execution flow.
    
    
    In terms of speed, the communication fusion fused together \c{six} points of synchronisation
    and greatly reduced constant factors in its runtime complexity by pushing
    replications into local PU operations. We can further improve on that with stream fusion.
    
  \subsection{Stream Fusioning}
    Stream Fusioning is our final step of optimisation. Applying it improves
    the normalisation and scaling. So, let's take a look at it.
    \begin{lstlisting}
let a0      = int2Double . headPS $ a 
  divisor = minusDouble (int2Double (lastPS a))
                . int2Double . headPS $ a
  gmax'   = int2Double $ gmax
  normScale = floorDouble
                  . (flip multDouble) gmax'
                  . (flip divDouble) divisor
                  . (flip minusDouble) a0
                  . int2Double
  gs = joinD . mapD (mapS normScale) . splitD $ a
     \end{lstlisting}
     Firstly, we can observe, how the replications have been completely removed! The
     constants are now first calculated and then used applied into the
     arithmetic functions. The arithmetic functions also have been merged together into
     a single function \c{normScale}! This function is now applied elementwise
     on each value in each of the local chunks of the entire histogram array.
     
     This was a result of inlining the local sequential functions like \c{multDoubleS}
     and subsequent stream fusion. For example, \c{multDoubleS} and \c{floorDoubleS} are defined as:\footnote{Note the similarity of stream fusion and communication fusion when it comes to the function definitions and rewrite rules.}
     \begin{lstlisting}
floorDoubleS :: Vector Double -> Vector Int
floorDoubleS = unstream . mapSt floorDouble . stream

multDoubleS :: Vector Double -> Vector Double -> Vector Double
multDoubleS as = unstream . zipWithSt multDouble (stream as) . stream
     \end{lstlisting}
     Inlining these definitions creates expressions of \c{unstream . stream}. Applying
     the "unstream/stream" rule and a few other rules similar to the previous
     communication fusion finally propagates the constants into the
     \c{normScale} closure.
     
     The transformation is - at least for our human minds - over now. The next
     section will give an overview of our results.
         
\section{Final Program}
  Summing up the parts, we end up with following optimised code for \ndpv:
  \begin{lstlisting}
type Image = PA (PA Int)
type Hist  = PA Int

hbalance :: Image -> Image
hbalance img =
let a :: Hist
    a = joinD
          . mapD (\(as,a) -> mapS (plusInt a) as)
          . propagateD plusInt 0
          . mapD (scanlS plusInt 0)
          . sparseToDenseD (plusInt gmax 1) 0
          . splitSparseD (plusInt gmax 1)
          . joinD
          . mapD tripletToATup2
          . segdSplitMerge 0
          . sortPS
          . concatPS
          $ img
    n :: Int
    n = lengthPS a
    
    a0, divisor, gmax' :: Double
    a0      = int2Double . headPS $ a
    divisor = minusDouble (int2Double (lastPS a))
                  . int2Double . headPS $ a
    gmax'   = int2Double gmax
    
    normScale :: Int -> Int
    normScale = floorDouble
                      . (flip multDouble) gmax'
                      . (flip divDouble) divisor
                      . (flip minusDouble) a
                      . int2Double
      
    gs :: Hist
    gs = joinD . mapD (mapS normScale) . splitD $ a
    
in unconcatPS img
     . indexPL
     . concatPS
     $ img
  \end{lstlisting} %  (expandPS img gs)
  On the surface - the algorithm works quite similar to a direct
  implementation of \ndpn.
  First the histogram is calculated (lines 18 to 11) and accumulated (lines 10 - 7).
  Then the constants \c{a0},\c{divisor} and \c{gmax'} are calculated globally and distributed
  to each PU (lines 22 to 33). After that, each PU applies the normalisation and scaling transformations(line 36).
  The mapping array \c{gs} is then finally used to map
  each gray tone to its new value (lines 41 to 38). It happens by using the
  flat representation of the nested image.
  
  All in all, \ndpv offers a few advantages:
  \begin{itemize}
    \item A decreased number of communication and synchronication points.
    \item Flat data structures and flat operations further decrease constant overhead factors 
    \item Distributed optimal-complexity prefix-sum, groupP and sortP
    \item We implemented normalisation and scaling separately, but inlining
            and optimisation still was able to fuse them together. We didn't
            manually need to fuse it - as we had to do in \man
    \item After writing \ndpn, there is no more work involved for the programer in generating this optimised code.
  \end{itemize}
  Having transformed \ndpn into \ndpv, we are now ready to give a complexity analysis thereof.
  
\section{Complexities}
  The complexity analysis for work and depths remains similar to that of \ndpn.
  After all, both are the same algorithm. \ndpv is only better at it's constant factors.
  Let $n = |img| = w\cdot h$ then we can calculate its complexity:
  \begin{equation}
  \begin{split}
  \W(w \times h,gmax)
        & = \W(hist) + \W(accu) + \W(gs) + \comment{\W(expandPS) +} \W(img') \\
        & = O( \max(n \log n, gmax) + gmax + gmax \comment{ + 1} + n) \\
        & \in O(\max(n \log n, gmax)) \\
  \D(w \times h,gmax)
      & = \max \{ hist, accu, gs\ceomment{, expandPS}, img'\} \\
      & = \max \{\log n, \log gmax \} \\
      & \in O(\log \max(n,gmax)) \\
  \end{split}
  \end{equation}
  The use of the plentyful new functions doesn't change the overall situation compared to \ndpn.
  We end up with the same work and depth complexities as before. Being at $O(\max(n \log n,gmax))$ in work
  the histogram calculation remains the most expensive when executed on a single processor.
  With increasing number of processors, the various logarithmic depth
  operations in the (accumulated) histogram calculation becomes the most expensive.
  The take logarithmic time in the number of gray tones \c{gmax} or the number
  of pixels \c{n} - depending on which one is greater.
  A overview of all the functions an their complexities can be found in the table \ref{complexities_ndpv} in the appendix.
  
  After implementing four algoruthmis and analysing their complexities, we are now ready
  to evaluate and compare them to each other. That is our goal for the next chapter.
  
  
  
  


\chapter{Evaluation (<5)}
  
% Vertrauen ist gut, Kontrolle ist besser.

This chapter summarises results of the the four implementations
\seq, \man, \ndpn and \npdv. First the complexities of
the implementations are given and discussed. Then
the pro and contra of each implementation are states.
Finally, this chapter end with a comparison of
sequential to parallel programming
and manual to compiler-optimised programming.

\section{Complexities}
  The complexities of all four programs are summarised in table
  \ref{table:allcomps}.
  
  \begin{table}[h]
    \caption{Work and Depth complexities}
    \label{table:allcomps}
    \centering
    \begin{tabular}{lll}
      \toprule
      program & work & depth \\
      \midrule
      \seq  & $n \log gmax + gmax$ & $n \log gmax + gmax$ \\
      \man  & $n \cdot gmax$ & $\log n$ \\
      \ndpn & $n \log n + gmax$ & $\log n + \log gmax$ \\
      \ndpn & $n \log n + gmax$ & $\log n + \log gmax$ \\
    \end{tabular}
  \end{table}
  One can make multiple observations.
  
  First, 
  
  pman better in parallel? pndpv better in work
  
  Second, Pnest and Pvect equal in work and depth. Only better
  constant factor. "The compiler cannot do the impossible".
  
  Third. sequential has the fastest work, but is purely sequential.
  (Parallel programs have an overhead).

  Different ratios of gmax and n make differnt algorithms faster. (binary tree of desicions)
  
\section{\seq}
  ...

\section{\man}
  ...

\section{\pndn and \ndpv}
  
  
  
\paragraph{Speedup in Parallisation}
  ...

\paragraph{\man vs \ndpn and \ndpv}
  Before moving to the next chapter - one shall be reminded that
  \man involved much manual work. It was not a direct translation
  of the algorithms description. It, especially requires
  the subsequent algorithms to use a flat image representation.
  \footnote{Unless one wraps \man with (un-)flattening operations.
  However, that approach is a manual replication of the
  flattening approach used in NDP.}



\chapter{Conclusion (<5)}
  
  % Zitat?

  ...
  \section{Effectiveness}
    ...
  \section{Related Work}
    Other work on optimisation in parallel functional programming (in haskell)
  \section{Future Work}
    ...
    "Alternate Algorithms"
      Rose Tree Data strcture -> Interesting, Custom Types, too compliated
      Djikstra on Image -> Too regular, huge for manual vectorization
        Connected Components Labeling -> huge for manual vectorization
        Contour Detection -> Too fast and simple to benefit from parallization
    "Distributed NDP"
      ...
  \section{Retrospection}
    ...
  \section{Final words}
    ...
    


%% appendix if used
%%\appendix
%%\typeout{===== File: appendix}
%%
\section*{SparseToDensePS}
  \label{lst:sparsetodenseps}
  \begin{lstlisting}
sparseToDensePS :: Int -> a -> PA (Int,a) -> PA a
sparseToDensePS size z ps = 
  joinD
  . sparseToDenseD size z
  . splitSparseD size
  $ ps

splitSparseD :: Int -> PA (Int,a) -> Dist (PA (Int,a))
sparseToDenseD :: Int -> a -> Dist (PA (Int,a)) -> Dist (PA a)
  \end{lstlisting}
  SparseToDensePS converts a sparse array of index-value-pairs
  to a dense array, where the elements are inserted in the appropriate
  indices. Unspecified indices are given the default value \c{z}.

  The function operates by first splitting/distributing the sparse array
  to the various PUs. It does that in such a way, that if all PUs were to
  hold a chunk of an array of lenght \c{size}, then each PU would get
  those index-value-pairs for which it would be responsible for on the
  dense array. This approach enables the second step to be purely local.
  The second step converts each local chunk of the sparse array to its corresponding
  local chunk of the dense array.

  Further analysis reveals complexities $\W(z,ps) \in O(z + length(ps))$
  and $\D(z,ps) \in O(1)$.


\chapter*{Appendix}
\addcontentsline{toc}{chapter}{Appendix}
  %
\section*{SparseToDensePS}
  \label{lst:sparsetodenseps}
  \begin{lstlisting}
sparseToDensePS :: Int -> a -> PA (Int,a) -> PA a
sparseToDensePS size z ps = 
  joinD
  . sparseToDenseD size z
  . splitSparseD size
  $ ps

splitSparseD :: Int -> PA (Int,a) -> Dist (PA (Int,a))
sparseToDenseD :: Int -> a -> Dist (PA (Int,a)) -> Dist (PA a)
  \end{lstlisting}
  SparseToDensePS converts a sparse array of index-value-pairs
  to a dense array, where the elements are inserted in the appropriate
  indices. Unspecified indices are given the default value \c{z}.

  The function operates by first splitting/distributing the sparse array
  to the various PUs. It does that in such a way, that if all PUs were to
  hold a chunk of an array of lenght \c{size}, then each PU would get
  those index-value-pairs for which it would be responsible for on the
  dense array. This approach enables the second step to be purely local.
  The second step converts each local chunk of the sparse array to its corresponding
  local chunk of the dense array.

  Further analysis reveals complexities $\W(z,ps) \in O(z + length(ps))$
  and $\D(z,ps) \in O(1)$.


% bibliography and other stuff
\backmatter

\typeout{===== Section: literature}
%% read the documentation for customizing the style
\bibliographystyle{apalike}
\bibliography{thesis}

%\typeout{===== Section: nomenclature}
%% uncomment if a TOC entry is needed
\addcontentsline{toc}{chapter}{Glossary}
\renewcommand{\nomname}{Glossary}
\markboth{\nomname}{\nomname} %% see nomencl doc, page 9, section 4.1
\printnomenclature

\clearpage


%% index
\typeout{===== Section: index}
\printindex

\HAWasurency

\end{document}
