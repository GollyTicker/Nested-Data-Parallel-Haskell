
\section*{SparseToDensePS}
  \begin{lstlisting}
    sparseToDensePS :: Int -> a -> PA (Int,a) -> PA a
    sparseToDensePS size z ps = 
      joinD
      . sparseToDenseD size z
      . splitSparseD size
      $ ps

    splitSparseD :: Int -> PA (Int,a) -> Dist (PA (Int,a))
    sparseToDenseD :: Int -> a -> Dist (PA (Int,a)) -> Dist (PA a)
  \end{lstlisting}
  SparseToDensePS converts a sparse array of index-value-pairs
  to a dense array, where the elements are inserted in the appropriate
  indices. Unspecified indices are given the default value \c{z}.

  The function operates by first spliting/distributing the sparse array
  to the various PUs. It does that in such a way, that if all PUs were to
  hold a chunk of an array of lengh \c{size}, then each PU would get
  those index-value-pairs for which it would be responsible for on the
  dense array. This approach enables the second step to be purely local.
  The second step converts each local chunk of the sparse array to its corresponding
  local chunk of the dense array.

  Further analysis reveals complexities $\W(z,ps) \in O(z + length(ps))$
  and $\D(z,ps) \in O(1)$.
  
\section*{Complexities for \ndpv}  
      
    \subsection*{Complexity table}
      A table of complexities is given in \ref{complexities_ndpv}.
      \begin{table}[h]
        \caption{Complexities for \ndpv}
        \label{complexities_ndpv}
        \begin{tabular}{lll}
          \toprule
          function or variable &      O(W)           & O(D) \\
          \midrule
          hbalance        & \max(n \log n, gmax)& \log \max(n, gmax) \\
          \midrule
          hist            & \max(n \log n, gmax)& \log n \\
          concatPS        & 1                   & 1 \\
          sortPS          & $n \log n$          & $\log n$ \\
          segdSplitMerge  & $n \log n$          & $\log n$ \\
          mapD tripletToATup2  & gmax           & 1 \\
          joinD xs        & gmax                & 1 \\
          \midrule
          accu            & gmax                & \log gmax \\
          sparseToDenseD  & gmax                & 1 \\
          splitSparseD    & gmax                & 1 \\
          mapD scanlS     & gmax                & 1 \\
          propagateD      & 1                   & $\log gmax$ \\
          mapD (mapS plusInt) & gmax            & 1 \\
          joinD           & gmax                & 1 \\
          \midrule
          as              & gmax                & 1 \\
          splitD          & gmax                & 1 \\
          mapD normScale  & gmax                & 1 \\
          joinD           & gmax                & 1 \\
          %\midrule
          %expandPS  & 1  & 1 \\
          \midrule
          img' (the new image)     & n                   & 1 \\
          concatPS        & 1                   & 1 \\
          indexPL         & n                   & 1 \\
          unconcatPS      & 1                   & 1 \\
        \end{tabular}
      \end{table}
      
    \subsection*{Functions/Variables}

    \subsubsection*{Hbalance}
      Let $n = |img| = w\cdot h$
      \begin{equation}
      \begin{split}
      \W(w \times h,gmax)
            & = \W(hist) + \W(accu) + \W(gs) + \comment{\W(expandPS)} + \W(img') \\
            & = O( \max(n \log n, gmax) + gmax + gmax \comment{ + 1} + n) \\
            & \in O(\max(n \log n, gmax)) \\
      \D(w \times h,gmax)
          & = \max \{ hist, accu, gs\comment{, expandPS}, img'\} \\
          & = \max \{\log n, \log gmax \} \\
          & \in O(\log \max(n,gmax)) \\
      \end{split}
      \end{equation}
        
    \subsubsection*{Histogram}
      Let $n = |img| = w\cdot h$
      \begin{equation}
      \begin{split}
      \W(w \times h)
            & = \W(concatPS) + \W(sortPS) + \W(segdSplitMerge) \\
            & + \W(mapD,tripletToATup2) + \W(joinD) \\
            & = 1 + n \log n + n \log n + gmax + gmax \\
            & = 1 + 2 (n \log n + gmax) \\
            & \in O(\max(n \log n, gmax)) \\
      \D(w \times h)
            & = \max \{ concatPS,sortPS,...,joinD\} \\
            & \in O(\log n) \\
      \end{split}
      \end{equation}
    
    \subsubsection*{Accumulated histogram}
      \begin{equation}
      \begin{split}
      \W(gmax)
            & = \W(splitSparseD) + \W(sparseToDenseD) + \W(mapD,scanlS) \\
            &     + \W(propagateD) + \W(mapD,mapS,plusInt) + \W(joinD) \\
            & = gmax + gmax + gmax + 1 + gmax + gmax\\
            & = 1 + 5 \cdot gmax \\
            & \in O(gmax) \\
      \D(gmax)
          & = \max \{ splitSparseD, sparseToDenseD, (mapD,scanlS),...\} \\
          & \in O(\log gmax)
      \end{split}
      \end{equation}
    
    \subsubsection*{Normalisation and Scaling}
      \begin{equation}
      \begin{split}
      \W(gmax)
            & = \W(joinD) + \W(mapD,normScale) + \W(splitD) \\
            & = gmax + gmax + gmax \\
            & = 3 \cdot gmax \\
            & \in O(gmax) \\
      \D(gmax)
          & = \max \{ joinD,(mapD,normScale),splitD\} \\
          & \in O(1)
      \end{split}
      \end{equation}
    \comment{
      \subsubsection*{expandPS}
        \c{expandPS} is a function which, if executed strictly
        and literally would have a work complexity of $n \cdot gmax$.
        However, due to strong improvements in \cite{EffiVect2012Lipp}
        this replication is simply cached locally such that
        the work complexity for this case is negilible and
        can be simplified to $1$. The improvements were not
        introduces, since it unessesarily would lengthen this
        thesis without significant insights.
        
        \begin{equation}
        \begin{split}
        \W(gmax,w \times h)
              & \in O(1) \\
        \D(gmax, w \times h)
              & \in O(1) \\
        \end{split}
        \end{equation}
    }
    \subsubsection*{img'}
      \begin{equation}
      \begin{split}
      \W(gmax, w \times h)
            & = \W(unconcatPS) + \W(indexPL) + \W(concatPS) \\
            & = 1 + n + 1 \\
            & \in O(n) \\
      \D(gmax, w \times h)
          & = \max \{ unconcatPS, indexPL, concatPS \} \\
          & \in O(1)
      \end{split}
      \end{equation}
    

