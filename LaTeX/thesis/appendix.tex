
\section*{SparseToDensePS}
  \label{lst:sparsetodenseps}
  \begin{lstlisting}
sparseToDensePS :: Int -> a -> PA (Int,a) -> PA a
sparseToDensePS size z ps = 
  joinD
  . sparseToDenseD size z
  . splitSparseD size
  $ ps

splitSparseD :: Int -> PA (Int,a) -> Dist (PA (Int,a))
sparseToDenseD :: Int -> a -> Dist (PA (Int,a)) -> Dist (PA a)
  \end{lstlisting}
  SparseToDensePS converts a sparse array of index-value-pairs
  to a dense array, where the elements are inserted in the appropriate
  indices. Unspecified indices are given the default value \c{z}.

  The function operates by first splitting/distributing the sparse array
  to the various PUs. It does that in such a way, that if all PUs were to
  hold a chunk of an array of lenght \c{size}, then each PU would get
  those index-value-pairs for which it would be responsible for on the
  dense array. This approach enables the second step to be purely local.
  The second step converts each local chunk of the sparse array to its corresponding
  local chunk of the dense array.

  Further analysis reveals complexities $\W(z,ps) \in O(z + length(ps))$
  and $\D(z,ps) \in O(1)$.
