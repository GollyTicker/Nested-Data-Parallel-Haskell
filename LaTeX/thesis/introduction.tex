

\section{Context}
  In the past few decades the raw processing power of CPUs has
  not increased by any significant amount.
  Instead, we are presented with hardware with an
  exponentially increasing number of processors
  - each of them as fast as the previous generation.
  Parallel COmputing is concerned with the use of multiple
  processing units to increase performance.
  Functional programming languages are
  particularly successfull in parallel programming
  due to their immutability and cache friendly-ness
  
  Research and Development in this field is generally referred to as Parallel Computing.
  For my thesis, I have further reduced the context to the programming language Haskell and Image Processing. An explanation is given next.
    
  
  A few questions arise:
  
  !Context like in the expose!
  
  Why Image Processing? Simply interesting topic. Using Histogram Balancing
  
  
  Focus on practical functional programming.
    Why Haskell?
    many researchers do their work in the haskell community,
      on of them will be used here - that is Nested Data Parallel Haskell
    I enjoyed haskell the most and learned many concepts there. (like in expose)
    
    Results can (and are) applied to other functional programming
    lanuguages. cite? 
    
    
    Constant Factors

\section{Goal}

  To answer these questions, an image processing algorithm
  is chosen and implemented in four variations.
  
  \ac $:=$ \algo, the conceptual image processing algorithm
  
  \seq $:=$ An ordinary Haskell implementation of \ac

  \man $:=$ A manually-parallelized implementation of \ac
  
  \ndpn $:=$ A Nested-Data-Parallel implementation of \ac

  \ndpv $:=$ The compiler-transformed implementation of \ac from \ndpn
        

  Leading question. Subquestion.
  Formulating the approach
  
  Formulate Concept/Algorithm of Histogram Balanding as $A$.

\section{Structure}
  Describe what each chapter does

